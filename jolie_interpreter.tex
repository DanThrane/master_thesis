% Status: draft

\section{The Jolie Engine and Interpreter}

\begin{itemize}
\item Values and types
\item Phases of the compiler
\item Anything we need to explain the changes that were made
\end{itemize}

% ---

In this section the internals of the Jolie engine will be introduced. This
should give the reader the necessary knowledge to understand the changes made
to support a module and package system for Jolie.

At the core of the Jolie engine is the interpreter. Each interpreter is
responsible for parsing and executing a Jolie program. A single engine may run
several instances of the interpreter, this is most commonly the case when
embedding several other Jolie services.

A simplified view of the Jolie interpreter's pipeline can be seen in Figure
\ref{fig:jolie_pipeline}.

\begin{listing}[H]
\begin{minted}{text}
                    +--------------------------+
                    |  Intent parsing          |
                    +--------------------------+
                                  |
                                  v
                    +--------------------------+
                    |  Jolie code parsing      |
                    +--------------------------+
                                  |
                                  v
                    +--------------------------+
                    |  Semantic verifier       |
                    +--------------------------+
                                  |
                                  v
                    +--------------------------+
                    |  Interpretation tree     |
                    +--------------------------+
                                  |
                                  v
                    +--------------------------+
                    |  Init process            |
                    +--------------------------+
                                  |
                                  v
                    +--------------------------+
                    |  Main process            |
                    +--------------------------+
\end{minted}
\caption{A simplified view of the Jolie interpreter pipeline}
\label{fig:jolie_pipeline}
\end{listing}

The first phase of the interpreter is parsing the intent. In this phase we
essentially figure out why the interpreter has been created, and what actions
it should perform.

The Jolie engine can be invoked from the command-line, the command-line is the
source of the intent which starts the first interpreter. The syntax for the
command line is (roughly) as follows: \mintinline{bash}{jolie [commands and
options] <program> [program arguments]}. The options passed change the
overall behaviour of the engine, and all interpreter instances share these.
Commands and program arguments, however, only belong to the interpreter that
they were originally passed.

The intent parsing phase is also responsible for locating and retrieving the
program used. This is passed to the program parser. The program parser is the
second phase, and is responsible for creating the Abstract Syntax Tree (AST)
which represents the input program. The parser will produce only a single root
node, namely the \verb!Program! node. As a consequence of this, any file which
is included is semantically identical to copying and pasting the source code of
that file into the original file, in place of the include statement.

The third phase traverses the AST to make sure that it is semantically valid.
This weeds out programs which are syntactically correct, but do not make sense.
The amount of work done in this phase is somewhat limited, given the otherwise
dynamic nature of the language.

Given a semantically valid AST the interpreter is ready to build the
interpretation tree. The interpretation tree contains new nodes, known as
processes. Each of these processes can be run, to execute the correct
behaviour.

Once the interpretation tree is build, we're ready to execute the actual code
(which lives in the interpretation tree). First the init block is run, followed
by the main block.
