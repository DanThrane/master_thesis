\section{The JPM Cache}

The \cache of JPM is responsible for downloading packages, and maintaining the
packages in a local cache.

The local cache acts as a database of known packages. This database contains
both the information needed to know about the database, as well as the binary
containing the package data.

The local cache consists of previously downloaded packages, along with a
locally kept database which matches the type that the \registry uses. The
database kept also track the origin \registry, since the \cache will
potentially download from multiple. The key used for this is the location URI
of each \registry. The name cannot be used, since these differ between each
package.

The primary operation that the \cache exposes is the \txtl{installDependency}
operation. This operation will install a dependency directly into a package.
JPM assumes that dependencies are installed under
\txtl{<pkgRoot>/jpm_packages/<depName>}. Having packages stored under a common
folder makes it easy to locate. This is, for example, useful when needing to
filter out the packages. Common places for this would be version control, or
from JPM when publishing a package.

The \cache will first attempt to retrieve the package from its local database,
if not found it will forward the request to the registry. In order to
forward the request, the operation is fed both the registry and
authentication token.

When a new package is received from a \registry, two parts will be returned:
the binary itself, along with the checksum that the \registry believes the
package to have. Checksums are both checked when initially receiving the
package, but also at every install from the cache. This helps prevent initial
corruption and further protects from the cache itself being corrupted. A more
thorough discussion of this feature can be found in Section
\ref{sec:integrity_check}.

The \cache will maintain a \regdb just like the \registry component does. When
new entries are loaded into the \cache, it will updated the \regdb in a similar
fashion. As a result the \cache is technically capable of performing the exact
same responsibilities as the \registry. This could for example allow for fully
offline downloads assuming the data is already loaded into the cache.

%\begin{listing}[H]
%\begin{minted}{text}
%+---------------+   +----------------+
%| packages      |   | checksum       |
%+---------------+   +----------------+
%       ^                    ^
%       |                    |
%       |    +---------------+
%       |    |
%+---------------+   +----------------+
%| cache         |-->| registry       |
%+---------------+   +----------------+
%       |
%       | (local db)
%       v
%+---------------+
%| registry-db   |
%+---------------+
%\end{minted}
%\caption{}
%\label{lst:}
%\end{listing}
