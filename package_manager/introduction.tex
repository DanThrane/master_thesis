\section{Introduction}

TODO This is just an introduction to the CLI. Re-write to match the package
manager overall

The command line application serves as the user interface to JPM. The
application is, perhaps not unsurprisingly, named \mintinline{text}{jpm}. The
tool will be used in several examples.

The command line application is responsible for displaying a more user friendly
interface to inner workings of JPM. The tool will perform almost not work by
itself, but will instead delegate this to the back end (See Section TODO).

When first running the tool, the user will be welcomed with the following
message:

\begin{minted}{text}
JPM - The Jolie Package Manager
Version 1.0.0

Usage: jpm <COMMAND> <COMMAND-ARGUMENTS>

Command specific help: jpm help <COMMAND>

Available commands:
-------------------
init           Initializes a repository
search         Searches repositories for a package
install        Install dependencies
publish        Publish this package
start          Start this package.

[ Remaining commands removed from snippet ]
\end{minted}

As clearly visible form this snippet, for the tool to do any work we must first
give it something to do via a command. The commands that JPM understands almost
directly mirror the functionality provided by the back end. To use JPM to
create a new package, the user must simply use the \mintinline{text}{init}
command. This will display a prompt, guiding the user through the mandatory
field, and automatically create a package with the required structure. This
is shown in Listing \ref{lst:jpm_init}.

\begin{listing}[H]
\begin{minted}{text}
$ jpm init
Package name
------------
> my-package

Package description
-------------------
> This is my package

Author: [Format: name <email> (homepage)]
-----------------------------------------
> Dan Sebastian Thrane <dathr12@student.sdu.dk> (github.com/DanThrane)

Private package? [Y/n]
----------------------
> n

$ cat my-package/package.json | json name
my-package
\end{minted}

\caption{The \mintinline{text}{jpm} tool provides a user interface for common
    tasks. In this example, creating a new package.}

\label{lst:jpm_init}

\end{listing}



