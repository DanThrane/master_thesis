\section{Introduction}

The Jolie Package Manager is build to facilitate the use of packages, which
were introduced in the previous chapter. The package manager provides a variety
of tools, these can roughly be divided into three categories:

The command line application serves as the user interface to JPM. The
application is, perhaps not unsurprisingly, named \mintinline{text}{jpm}. The
tool will be used in several examples.

\begin{enumerate}
\item \textbf{Package management}
    \begin{itemize}
        \item Installing packages (Section \ref{sec:cache})
        \item Publishing packages (Section \ref{sec:registry})
        \item Upgrading packages (Section \ref{sec:versions} and
                \ref{sec:lockfiles})
        \item Searching for packages (Section \ref{sec:regdb})
    \end{itemize}
\item \textbf{Account management (Section \ref{sec:security})}
    \begin{itemize}
        \item Login/logout with registries
        \item User management
        \item Team management
    \end{itemize}
\item \textbf{Helper scripts}
    \begin{itemize}
        \item Creating a new package (Section \ref{sec:jpm_basic_example})
        \item Starting a package (Section \ref{sec:jpm_basic_example})
        \item Interacting with the cache (Section \ref{sec:cache})
    \end{itemize}
\end{enumerate}

These responsibilities are services by the package manager ecosystem which were
briefly covered in Section \ref{sec:jpm_simple_arch}. Many of the features this
ecosystem provides were also covered in Chapter \ref{cha:packages}. In this
chapter we will cover the final details of this ecosystem.

