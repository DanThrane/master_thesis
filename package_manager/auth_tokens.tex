\section{Authentication Tokens}

In Section \ref{sec:authorization} we covered the authorization framework that
JPM uses. When a user successfully authenticates the authorization framework
will provide an authentication token proving the user's identity.

For user convenience the JPM system does not require a valid username and
password combination for every privileged operation. To avoid this, JPM will
store the authentication token returned by the \registry. These are stored in a
file on the file-system.

TODO I can't write smart things right now. Words no good :-(

The authentication token (or session token) is saved instead of the username
and password.  Since if an attacker were to gain access to both the username
and password, she would be able to gain full control over the user, this
includes being able to generate new authentication tokens.

If an attacker instead were to gain only the authentication token, the damage
is not as bad. The attacker would still be able to pose as the user, but only
while the authentication token is valid.

Several steps are taken to mitigate the damage that can be done in case of
session hi-jacking. TODO I just called a session, without defining it.

% TODO Not sure if this needs to be implemented, but be careful when
% discussing. Don't make it sound like something is implemented if it isn't.
% Should make clear if something is used or not.

% We partially do this. It is possible, but needs to be implemented on the
% registry side of things. Would be an easy change.

Session timeout

% We do this

Manual session invalidation (logging out)

% We support this. Haven't defined any actions to be critical though.

Re-authentication before critical actions


\url{https://www.owasp.org/index.php/Authentication_Cheat_Sheet}

\url{https://www.owasp.org/images/0/08/OWASP_SCP_Quick_Reference_Guide_v2.pdf}
