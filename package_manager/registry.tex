\section{Registry}
\label{sec:registry}

In this section we will briefly cover the responsibilities of JPM registries.
For the most part we will cover the specific features it provides in other
section and make references to them, when relevant.

The primary responsibility of a JPM registry is to manage packages and serve
them to clients who request them. When a developer publishes a packages then
this is done to a particular package registry. When a developer downloads a
package this is done from a registry. Almost anytime the \txtl{jpm} tool needs
to do anything it must contact a registry for information.

Many registries can exist in the JPM ecosystem. A default registry, the public
registry\footnote{As of writing this thesis, the module system and package
    system has not yet been merged into the Jolie language. As a result, when
    referring to the public registry, we're referring to a registry which might
    exist in the future. The software required to run a registry is, however,
    implemented, as described in this thesis. A public registry does
    not exist yet.}, is made for sharing open-source packages with the Jolie
community. Operations in the \txtl{jpm} tool that needs to speak to a registry
will default to this registry.

Having multiple registries is beneficial for example if an organization wishes
to share packages, but only wishes to share them internally within the
organization. The JPM registry, and all other packages in the JPM ecosystem,
are JPM packages. They are available from the public registry. This makes it
relatively easy to host your own registry, since the process of this is the
exact same as any other Jolie package.

The \registry package also provides user and team management. Different rights
may be assigned to these users, which can control who may perform certain
actions. This is needed for controlling who's allowed to publish updates for a
package. This also allows for a registry to be configured to only allow users
from within an organization to use it. These features are covered by the
\security package which is described in Section \ref{sec:security}.

Keeping track of package information is delegated to the \regdb package. We
cover this package in Section \ref{sec:regdb}.
