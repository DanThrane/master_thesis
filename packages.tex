% Status: Draft

\section{Packages}

A Jolie Package is an extension of a Jolie Module. Recall that a Jolie Module
was defined as a collection of resources, a name, and optionally an entry-point
for the module. A package extends this concept by adding information required
for package management. % TODO Last sentence sucks

A Jolie Package is described by a package manifest. The package manifest is a JSON file, which is always placed at the root of the package, and must be
called \verb!package.json!. The fixed location allows for the package manager
to easily identify a package. The JSON format was chosen as it plain-text, and
easy to both read and write for both humans and machines.

In listing \ref{lst:simple_manifest} we show a simple package manifest. This
manifest showcases the most important features of the manifest.A complete
specification of the package manifest format can be seen in Appendix A. The
service that this manifest describes is shown in figure \ref{fig:simple_calc}.

\begin{listing}[H]
\begin{minted}{json}
{
    "name": "calculator",
    "main": "calc.ol",
    "description": "A simple calculator service",
    "authors": ["Dan Sebastian Thrane <dathr12@student.sdu.dk>"],
    "license" "MIT",
    "interfaceDependencies": [
        { "name": "math", "version": "1.0.0" }
    ],
    "dependencies": [
        { "name": "sum", "version": "1.2.X" },
        { "name": "multiplication", "version": "2.1.0" }
    ]
}
\end{minted}
\caption{A Simple Package Manifest}
\label{lst:simple_manifest}
\end{listing}

\begin{listing}[H]
\begin{minted}[linenos=false]{text}
                +--------------+     +-------+
                |              | ==> |  sum  |
                |              |     +-------+
                |  calculator  |
                |              |     +------------------+
                |              | ==> |  multiplication  |
                +--------------+     +------------------+
\end{minted}
\caption{A Calculator Service}
\label{fig:simple_calc}
\end{listing}

Lines 2-3 take care of the module definition. The remaining attributes,
however, are entirely unique to packages. Some attributes included in the
manifest are there for indexing and discoverability purposes, examples of such
attributes are shown in lines 4-6. The rest of the document describes the
dependencies of this package.

A JPM dependency is defined as a ``code dependency''. What this means is that
the dependencies listed, should only be packages that we depend on for code.
This might be different from the typical definition in microservices. For
example, one might state that the calculator also has a dependency on the
client who speaks to it. In JPM however, we will not need to list the client,
as we do not depend on any code that the client has (in fact we know nothing
about how the client works).

JPM deals with two different types of dependencies: interface dependencies, and
ordinary dependencies. These two serve similar, but slightly different
purposes, and all depends on how the dependency is to be used.

An ordinary dependency should be used if we wish to use the code of some other
service, and want to option of embedding it. This will tell JPM to download
both ordinary dependencies, as well as interface dependencies.

Interface dependencies are instead dependencies that are only required to
interface with the package itself, and dependencies we need to interface with
others. An interface dependency will only cause us to download the interface
dependencies of each sub-dependency.

% TODO Need an example and algorithm for this stuff!

The interface dependency type isn't common in other package managers, but
having multiple types of dependencies is fairy common.  For example, a package
manager might provide dependencies which are only used during testing, we see
this in Maven, or it might provide dependencies only used during development,
seen in NPM.

