\section{Modules}

A module in Jolie is defined as a project root directory, and optionally an
entry-point. Modules are uniquely identified by a name.

The Jolie engine needs to know about these modules. The engine is informed
about these modules from the intent as options which starts the engine
(command-line arguments). This information is collected in the ``intent
parsing'' phase, and is made available to any later phase that might need it.
This intent is passed as an option, recall that options are shared among all
interpreter instances inside of an engine. As a result any embedded service
will also be aware of the same modules.

As an example, we may inform the engine about a module called ``foo'', which
has its source code placed in \verb!/packages/foo!  with an entry-point in
\verb!/packages/foo/main.ol! we would write: \mintinline{text}{--mod
foo,/packages/foo,main.ol}. This implementation technique is very close to
how, for example, you would add a JAR file to the classpath of a Java program.
% TODO Very informal argument

The result of this decision is that quite a lot of additional options may need
to be passed to the engine. However this choice was purposely chosen, it
is left for another tool to make this job easier. In this case a package
manager is expected to take the heavy lifting, and figure out which modules
exists. Requiring complete knowledge to be passed to the engine, allows
for more freedom in how these tools are implemented, and another implementation
strategy, than the one provided in the package manager, could be created
without any changes to the language infrastructure. See Section TODO about how
the package manager passes this information to the engine.

To allow for better organization, a new type of include has been added to the
language. These include allows the developer to include from a particular
package. The syntax of this include is shown in Listing \ref{lst:mod_include}.
When a package include is used, the search path will be altered, such that the
project root is changed to that of the package (as opposed to where the engine
was started). Any file included from within this package should perform its
includes relative to its own root, rather than the project root.

TODO Example

The new include syntax, along with native support for modules, allows for the
code to properly organized. With this a module may be placed inside of its own
directory, without any of the code having to be changed.

\begin{listing}[H]
\begin{minted}{java}
include "<file>" from "<module>"
\end{minted}
\caption{Extension to the include statement, made for module imports}
\label{lst:mod_include}
\end{listing}

\subsection{Include Algorithm for Jolie}

TODO To this date I still don't understand how the include algorithm works. Ask
Fabrizio about this.

\subsection{Module Include Algorithm}

TODO Module includes modify the search path. This also means the injection of
the include directory, which is used as a convention. This helps maintain
expected behaviour.

TODO In order to implement the root switching we use a stack.
