\section{Modules: Design and Implementation}

The module system of Jolie is made to facilitate better options for the
modularization of code. This is what allows for a package-manager to exist. We
discuss the package manager in detail in Chapter \ref{cha:packages} and
\ref{cha:pkg_mngr}. A Jolie module is described by the following three
properties:

\begin{enumerate}

    \item \textbf{A name.} For a single running Jolie engine, this will
    uniquely identify a module

    \item \textbf{A file path to the Jolie code-base.} We will refer to this as
    the project root or just root.

    \item \textbf{(Optional) The entry point to the Jolie code.} This is a file
    path given relatively to the project root.

\end{enumerate}

The Jolie engine needs to know about these modules. The engine is informed
about these modules from the intent, passed as options, which starts the
engine. This information is collected in the ``intent parsing'' phase and is
made available to any later phase that might need it.  Since the intent is
passed as an option, all interpreter instances inside the engine will know
about each module.  As a result, any embedded service will also be aware of the
same modules.

Consider a module ``foo'', with its source code placed at \txtl{/packages/foo}
and an entry-point at \txtl{/packages/foo/main.ol}. It is possible to inform a
Jolie interpreter about this module using
\txtl{--mod foo,/packages/foo,main.ol}.

As a result, a lot of additional options must be passed to the engine. In this
case, a package manager is expected to take the heavy lifting and figure out
which modules exist.  This allows for more freedom in how these tools are
implemented. If another approach were to be found later, then a new
implementation could simply be created without breaking the language.

A new include primitive has been added, a module include. This has been created
to allow for better internal organization. This primitive allows the developer
to include files from a particular module. The new primitive is shown in
Listing \ref{lst:mod_include}. When a module include is used, the search path
will be altered, such that the project root is changed to that of the package.
As opposed to the original include primitive, which would look for files
relative to the current working directory.

The new include primitive, along with native support for modules, allows for
the code to be properly organized. With this, a module may be placed inside of
its own directory, without any code has to change.

\begin{listing}[H]
\begin{minted}{jolie}
include "<file>" from "<module>"
\end{minted}
\caption{Extension to the include statement, made for module imports}
\label{lst:mod_include}
\end{listing}

\subsection{Module Include Algorithm}

A module include is an extension of ordinary includes which specify which
module to perform the inclusion from, that is \joliel{include "file" from
"module"}.

The base case of such an include is simple. We make a ``context switch'' from
our working directory into the root of the module and start searching for the
file in that folder. This works quite well, however, consider the following
scenario.  From a client we perform a module include: \joliel{include
    "interface.iol" from "dependency"}.

\begin{listing}[H]
\begin{minted}{java}
/* /dependency/interface.iol */
include "types.iol"

interface Dependency { /* ... */ }

/* /dependency/types.iol */
type SomeRequest: void { /* ... */ }
\end{minted}

\caption{Module includes may cause more ordinary includes. We must make sure
    the switch in context is kept}

\label{lst:dependency_include}
\end{listing}

With the code in Listing \ref{lst:dependency_include}, the system would
incorrectly look for \txtl{types.iol} in the current working directory! To fix
this a stack is kept.  This stack gains an entry when a module include is
performed, this entry is popped off the stack once the contents of that include
has finished parsing.  The stack is then used to inject the module's root into
the search path for the ordinary includes.
