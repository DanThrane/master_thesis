\section{Examples}
\label{sec:module_examples}

In this section, we will introduce two examples which will showcase different
aspects of the Jolie module and configuration system. These examples will be
revisited again in Section \ref{sec:package_manager_examples} where they will
be shown in the context of packages.

\subsection{A Calculator System}

In this example, we will look at the calculator example shown in the
introductory chapters. The calculator system will be updated to work under the
new Jolie module system. We will also show how what was previous problems that
required code changes, can now be performed entirely in independent
configuration files.

Quickly recapping from previous chapters.  Figure \ref{fig:simple_calc} shows
the system architecture for the calculator system. The system contains three
services. The main point-of-contact for a client is the \txtl{calculator}
service. This service will delegate the ``hard'' work to the other two
services, that is \txtl{addition} and \txtl{multiplication}.

\begin{figure}[H]
\centering
\includegraphics[width=1.0\textwidth]{modules/simple_calc.eps}
\caption{System architecture for the ``calculator system''}
\label{fig:simple_calc}
\end{figure}

The file structure of our services contained a lot of duplication previously.
The file structure is shown in Listing \ref{lst:start_file_structure}. As we
can see from this there is quite a bit of duplication of files, and we cannot
simply copy the directories of our dependencies, as this would place the files
in the wrong places. We also had to resort to code files for distributing
default configuration profiles. Switching between embedding a service and
binding to it externally would also require significant changes.

\begin{listing}[H]

\begin{multicols}{3}

\begin{minted}[breaklines,fontsize=\footnotesize,frame=none]{text}
// multiplication
.
|-- multiplication.ol
|-- include
|   |-- mult_config.iol
|   `-- multiplication.iol
`-- numbers.iol

// numbers
.
|-- include
|   `-- numbers.iol
`-- main.ol
\end{minted}

\columnbreak

\begin{minted}[breaklines,fontsize=\footnotesize,frame=none]{text}
// calculator
.
|-- addition_config.iol
|-- addition.iol
|-- addition.ol
|-- calculator.ol
|-- include
|   `-- calculator.iol
|-- mult_config.iol
|-- multiplication.iol
|-- multiplication.ol
|-- numbers.iol
`-- numbers.ol
\end{minted}

\columnbreak

\begin{minted}[breaklines,fontsize=\footnotesize,frame=none]{text}
// addition
.
|-- include
|   |-- addition_config.iol
|   `-- addition.iol
`-- addition.ol
\end{minted}

\end{multicols}

\caption{Initial file structure of the calculator system}
\label{lst:start_file_structure}
\end{listing}

The first step is to update all includes which require files from other
modules, to use the new module includes. For example, \txtl{multiplication}
depends on the file \txtl{numbers.iol} from \txtl{numbers}.  Their includes
change from the old \joliel{include "numbers.iol"} to the new module include
\\\joliel{include "numbers.iol" from "numbers"}.

Using module includes provides a certain level of namespacing for file
names. This system will allow for two files to be named the same thing, without
causing problems, as long as they exist in separate packages. This allows us to
give files better names, and allow many rather useless prefixes to be deleted.
For example, the file \txtl{addition.iol} can be renamed to a more meaningful
name such as \txtl{interface.iol}. Causing includes to say \joliel{include
    "interface.iol" from "addition"}.

The next step is to convert the legacy configuration into the new configuration
format. The contents \txtl{mult_config.iol} previously contained constants for
configuring the input port and parameters of the service. The legacy
configuration is shown in Listing \ref{lst:mult_config_legacy}.

In order to convert ports, we must first decide which ports need to be
configured, and which fields should be configurable. In this case, we should
only configure our input port. Since the module makes no assumptions about this
service's input port, we can simply allow it all to be configurable.

For parameters we must determine the actual type each parameter can take. This
must be put in the \joliel{parameters} block. The complete transformation is
shown in Listing \ref{lst:mult_config_updated}.

\begin{listing}[H]

\begin{minted}{jolie}
constants {
    MULT_INPUT_LOCATION = "socket://localhost:12345",
    MULT_INPUT_PROTOCOL = sodep,
    MULT_MAX_INPUT = 1000,
    MULT_MIN_INPUT = 0,
    MULT_DEBUG = false
}
\end{minted}

\caption{Legacy configuration file for the \txtl{multiplication} module}

\label{lst:mult_config_legacy}

\end{listing}

\begin{listing}[H]
\begin{minted}{jolie}
inputPort Multiplication {
    Interfaces: IMultiplication
}

parameters {
    minInput: int,
    maxInput: int,
    debug: bool
}
\end{minted}
\caption{Updated configuration for the \txtl{multiplication} module}
\label{lst:mult_config_updated}
\end{listing}

We're not quite finished yet with the configuration. The original service
provided default configuration. A new directory named \txtl{conf} is created.
Inside which we can make as many defaults as we wish. For now, we create a file
named \txtl{default.col}. In Listing \ref{lst:mult_default_config} we show the
default configuration.  The inheritance feature allows us to easily create
profiles for various deployment configurations which a user of the module might
wish to use. The \joliel{"self-hosted"} profile would be ideal for hosting the
service as standalone, while the \joliel{"embedded"} and \joliel{"debug"}
profiles would be ideal for development.

\begin{listing}[H]
\begin{minted}{jolie}
profile "default" configures "multiplication" {
    minInput = 0,
    maxInput = 1000,
    debug = false
}

profile "self-hosted" configures "multiplication" extends "default" {
    inputPort Multiplication {
        Location: "socket://localhost:12345"
        Protocol: sodep
    }

    outputPort Numbers {
        Location: "socket://numbers.example.com:22000"
        Protocol: sodep
    }
}

profile "embedded" configures "multiplication" extends "default" {
    inputPort Multiplication {
        Location: "local"
        Protocol: sodep
    },

    outputPort Numbers embeds "numbers" with "default"
}

profile "debug" configures "multiplication" extends "embedded" {
    debug = true
}
\end{minted}

\caption{Providing defaults for a Jolie module. Using inheritance we can create
suitable profiles for typical deployment configurations.}

\label{lst:mult_default_config}

\end{listing}

At this point we can now clean-up in the directories of each module. The Jolie
module system technically allows for any organization of modules, but for now
we will simply put the source code of each module inside a directory called
\txtl{modules}. In this directory we will place a directory named the same as
the module, containing the complete source code of the modules.

After doing this, and renaming files to give them more meaningful names, we end
up with the file structure shown in listing \ref{lst:final_file_structure}.

\begin{listing}[H]

\begin{multicols}{3}

\begin{minted}[breaklines,fontsize=\footnotesize,frame=none]{text}
// multiplication
.
|-- conf
|   `-- default.col
|-- include
|   `-- interface.iol
|-- main.ol
`-- modules
    `-- numbers


// numbers
.
|-- include
|   `-- interface.iol
`-- main.ol
\end{minted}

\columnbreak

\begin{minted}[breaklines,fontsize=\footnotesize,frame=none]{text}
// calculator
.
|-- include
|   `-- interface.iol
|-- main.ol
`-- modules
    |-- addition
    |-- multiplication
    `-- numbers
\end{minted}

\columnbreak

\begin{minted}[breaklines,fontsize=\footnotesize,frame=none]{text}
// addition
.
|-- conf
|   `-- default.col
|-- include
|   `-- interface.iol
`-- main.ol
\end{minted}

\end{multicols}

\caption{Final file structure of the three services for the calculator system}
\label{lst:final_file_structure}
\end{listing}

This file structure, as we shall also discuss later, is much better for
potential package managers. Launching the service now requires a longer
command. Currently in order to launch this program the following would be
needed:

\begin{minted}{text}
dan@host:/calculator # jolie                            \
    --mod addition,modules/addition,main.ol             \
    --mod multiplication,modules/multiplication,main.ol \
    --mod numbers,modules/numbers,main.ol               \
    --mod calculator,.,main.ol                          \
    calculator.mod
\end{minted}

In contrast, this is quite a bit more than previously:

\begin{minted}{text}
dan@host:/calculator # jolie calculator.ol
\end{minted}

However, most of this disappears when using a package manager or similar tool.
In fact it will be as simple as:

\begin{minted}{text}
dan@host:/calculator # jpm start
\end{minted}

This is also similar to how, for example, the Java ecosystem ordinarily does
its additions of libraries. From the CLI this will require manually listing
every JAR dependency. However such projects are usually managed through more
advanced build tools, such as: Maven, Gradle or Ant.

The configuration system also helps significantly with changing between various
different deployment configuration, which switch between embedding services and
having them be external.

Before the difference between embedding a service, and not embedding a service
would require drastically different work. The first step here would be to make
sure that all files were included in the directory. Since there were no module
system, these files would have to live in the same directory as all the other
code, or alternatively rewrite all includes in the dependency.
Secondly it would require changing all relevant configuration, and inserting
an embedded section inside of our source code.

With the module system this will require absolutely no code changes. The
calculator module would simply have to define the output ports for
\txtl{addition} and \txtl{multiplication} as configurable. This is shown in
Listing \ref{lst:conf_deps}.

\begin{listing}[H]
\begin{minted}{jolie}
outputPort Addition         { Interfaces: IAddition         }
outputPort Multiplication   { Interfaces: IMultiplication   }
\end{minted}
\caption{Definitions of the output ports for \txtl{addition} and
    \txtl{multiplication}}
\label{lst:conf_deps}

\end{listing}

Creating configurations for embedding these and for externally binding is
trivial as shown in Listing \ref{lst:conf_trivial_embedding}. No changes to the
existing code is required at all. These profiles can safely use the
\joliel{"debug"} profiles created in the default configuration files of both
\txtl{addition} and \txtl{multiplication} since those are included by default.

\begin{listing}[H]
\begin{minted}{jolie}
profile "calculator-with-embedding" configures "calculator" {
    outputPort Addition embeds "addition" with "debug"
    outputPort Multiplication embeds "multiplication" with "debug"
}

profile "calculator-with-mixed" configures "calculator" {
    outputPort Addition embeds "addition" with "debug"
    outputPort Multiplication {
        Location: "socket://mult.example.com:12345"
        Protocol: sodep
    }
}
\end{minted}

\caption{Creating vastly different deployments no longer require changes to the
code. They can instead be solved entirely from configuration files}

\label{lst:conf_trivial_embedding}

\end{listing}

