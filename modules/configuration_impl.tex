\section{Implementing the Configuration Format}
\label{sec:col_impl}

The new interpreter pipeline is as follows, new phases marked in bold text:

\begin{enumerate}
\item Intent parsing
\item Jolie code parsing
\item \textbf{Configurator}
\item Semantic verifier
\item Interpretation tree builder
\item \textbf{Late verification}
\item Communication core init
\item Init \& main process
\end{enumerate}

The primary work occurs in the configurator service, but some of the
verification work, namely the parameters, are postponed until after the
interpretation tree has been build.

\subsection{The Configuration Phase}

The configuration phase builds upon the result of the intent and parsing phase.
From the intent phase, we receive information about modules and the selected
configuration unit. From the parsing phase we, obviously, get the AST.

The configuration phase works by modifying the AST to insert the configuration
gathered from the selected configuration unit. The processed AST is handed to
the later phases, and the remainder of the pipeline works mostly like it has
before.

\subsubsection*{Processing of Configuration Files}

\paragraph{Resolving configuration files}

The first step of the configurator is to gather a fully processed configuration
unit. This process starts by looking for the configuration file which was
passed from the intent. The selected configuration unit is passed with:
\\\txtl{--conf <unitName> <filePath>}.

If the file path is an absolute file path, then that file will always be used.
However if the file path is relative some searching is required. The base case
for resolving this is to look for the configuration file relative to the
current working directory. However this approach stops working once modules starts
embedding other modules in a static fashion.

Consider three modules \txtl{A} and \txtl{B}, \txtl{C}. Module \txtl{A} depends
on \txtl{B}, and \txtl{B} depends on \txtl{C}.

\begin{listing}[H]
\begin{minted}{jolie}
/* /a/main.ol */
outputPort B { Interfaces: BInterface }

/* /a/config.col */
profile "a-deployment" configures "a" {
    outputPort B embeds "b" with "b-deployment"
}

profile "b-deployment" configures "b" { /* ... */ }

/* /b/main.ol */
outputPort C { Interfaces: CInterface }

embedded {
    Jolie:
        "--conf c-deployment embedded.col c.mod"
}

/* /b/embedded.col */
profile "c-deployment" configures "c" { /* ... */ }
\end{minted}

\caption{Relevant source code for modules \txtl{A} and \txtl{B}}

\label{lst:need_for_parent}

\end{listing}

Relevant code is shown in Listing \ref{lst:need_for_parent}.  Assume that we
start module \txtl{A} with \\\txtl{--conf a-deployment config.col a.mod}.  The
problem arises because the fixed embedding of module \txtl{C} performed in
\txtl{B}. Because the files are resolved relative to the working directory in
which the engine was started, the engine would look for the configuration file
in module \txtl{A}. To fix this, every time an embedding is performed the
engine will implicitly add a \txtl{--parent} flag. This flag informs the new
interpreter who performed the embedding. If the parent flag has been given, the
configuration file will be found relative to that of the parent.

The working directory, throughout this, is \txtl{/a/}.  The following will
happen:

\begin{enumerate}

\item Start engine (from CLI) with: \txtl{--conf a-deployment config.col a.mod}
File resolved to \txtl{/a/config.col} because of relative path and no parent.

\item Configuration causes embedding of \txtl{b-deployment} with:
\\\txtl{--parent a --conf b-deployment /a/config.col b.mod}. File resolved to
\\\txtl{/a/config.col} because of absolute path.

\item Source embedding in \txtl{B} causes embedding of \txtl{c-deployment}
with: \\\txtl{--parent b --conf c-deployment embedded.col c.mod}. File resolved
to \\\txtl{/b/embedded.col} because of relative path with parent.

\end{enumerate}

\paragraph{Parsing of configuration files}

The resolved configuration file is passed to the configuration parser along
with known modules. The parser is written in the same style as the Jolie code
parser.  The parser is a hand-written recursive descent parser with one token
lookahead.

The parser directly outputs the AST. At the top level, the AST consists of
\javal{ConfigUnit}s which directly map to the configuration unit abstraction.
Each \javal{ConfigUnit} is namespaced under the module that they configure.
Internally each namespace maps the name of the \javal{ConfigUnit} to the actual
\javal{ConfigUnit}. The top-level node of the AST contains a dictionary mapping
between the name of modules and their namespaces. The Java type for this
mapping ends up being \javal{Map<String, Map<String, ConfigUnit>>}, this allows
for easy access of \javal{ConfigUnit}s when they are needed. This does however
also mean that duplicates must be caught during parsing since they would
otherwise just silently override previous entries. This would typically have
been performed in a later stage to keep the parser ``pure''. In this case, some
purity was sacrificed for some efficiency and ease of use.

The grammar of the configuration format closely follows the grammar of Jolie
code. As a result, the configuration format also re-uses AST nodes where
possible. As we will see later, this proved helpful when outputting a
configured AST. The configuration grammar only supports a subset of the
features that Jolie code does.

Recall from Section \ref{sec:conf_units} that each module may publish default
configuration units, which can be used by user-level configuration units. For
this reason, the parser will make special note of references to other
configuration units. Such references currently only appear in output port
embeddings, e.g., \\\joliel{outputPort Foo embeds "foo" with "a-foo-unit"}.

Once the initial file has been parsed, every single configuration file placed
in the ``conf'' directory of referenced modules are parsed and placed in the
same AST. Configuration files parsed during this phase may itself have
references to other modules, in that case these will be pulled in as well.  A
set of parsed defaults is kept to avoid parsing the same file multiple times
or potentially reach infinite loops.

\paragraph{Merging configuration files}

With the entire configuration tree in place, merging is performed to create a
single \javal{ConfigUnit} which represents the actual configuration of this
module. The merging occurs between a \txtl{child} and its \txtl{parent} unit
(that is, the one it extends, if any) and results in a \txtl{merged} unit. The
\txtl{parent} is always merged, via a recursive call, first. The first
\txtl{child} selected is the \txtl{ConfigUnit} matching the profile name and
package given by the intent.

For ports, the merging algorithm works by inserting all the ports of the
\txtl{child} into the \txtl{merged} unit. The ports of the \txtl{parent} are
then merged with the preliminary output of the \txtl{merged} unit. The
process of merging a \txtl{port} of the \txtl{parent} with the \txtl{merged}
unit is shown, in pseudo code, in Listing \ref{lst:merge_ports}. In short the
merging will favor the \txtl{child}, and the \txtl{parent} will only provide
values if the \txtl{child} doesn't.

\begin{listing}[H]
\begin{minted}[linenos=true]{text}
procedure merge(port: Port, merged: ConfigTree): Port {
    if (!merged.contains(port)) return port
    child = merged.getExisting(port)

    if (child.isComplete) return child
    // The child cannot be an embedding, since those cannot
    // be partial.

    for (property in Port.properties) {
        if (child[property] == null) {
            child[property] = merged[property]
        }
    }
    return child
}
\end{minted}

\caption{Pseudo code for merging a \txtl{parent}'s \txtl{port} into a
    \txtl{merged} configuration tree}

\label{lst:merge_ports}

\end{listing}

For parameters the same idea applies. However each parameter definition node in
the AST potentially only provides a partial value. For example the node
\joliel{foo.bar = 42}, where the type of \joliel{foo} is \joliel{string { .bar:
    int }}, only provides a partial definition of \joliel{foo}. For this
    reason, the parameter assignments are first added from the \txtl{parent}
    and then by the \txtl{child}. This allows the definitions of the
    \txtl{child} to override those of the \txtl{parent}.

\subsubsection*{Applying Configuration to the AST}

The \txtl{merged} unit from the previous section is then used along with the
AST corresponding to the Jolie program. The last phase of the configurator will
process the entire AST and output a new, fully configured, AST. This process
works by running a \txtl{process} function on every single top-level node. The
\txtl{process} function will inspect the node and return a list of replacement
nodes.

Throughout this process the configurator will also verify that no configuration
is performed on nodes which aren't configurable. This includes dynamic ports
and non-empty interface definitions.

\paragraph{Ports}

When a port node is encountered in the AST, the \txtl{merged} unit is searched
for an output port matching its name. If none is found, the node is returned as
it is. Otherwise the configuration proceeds. For input ports and non-embedding
output ports, the process is the same. The properties of the AST port and the
port represented in the configuration unit are compared one-by-one. If any
property is defined by both, an error is returned pointing to the AST node and
the place that the configuration took place. At the end, if no errors are
returned, a new port is returned with their properties merged together.

For output ports which embed another module, the process is slightly different.
The configurator will first verify the existence of the configuration unit by
looking it up in the configuration tree. If it is not found an error will be
returned. Otherwise, the configurator will output a new embedding node. The
embedding node will contain a reference back to the same configuration file the
current \txtl{ConfigUnit} comes from. When the embedding is performed the
configurator of the embedded service will lookup the correct configuration unit
from the same file. This will cause an addition parsing of the same file, but a
caching layer at the configuration parser could take care of this.  The
embedding process is demonstrated in Listing \ref{lst:embedding_output}.

\begin{listing}[H]
\begin{multicols}{2}

\begin{minted}[breaklines,fontsize=\footnotesize]{jolie}
embedded {
    Jolie:
        "--conf foo-profile <inputConfigFile> foo.mod" in Foo
}
\end{minted}

\columnbreak

\begin{minted}[breaklines,fontsize=\footnotesize]{jolie}
outputPort Foo embeds "foo" with "foo-profile"
\end{minted}

\end{multicols}

\caption{AST nodes generated, shown as code (left), by configuration (right)
    for an output port containing an embedding}

\label{lst:embedding_output}

\end{listing}

\paragraph{Interface Rebinding}

Just like ports, when an interface definition is found, the \txtl{merged} unit
is searched for a matching interface definition. Assuming no errors, the
configurator must now replace the dummy interface definition with the real
interface definition, as the configuration specifies.

Quickly recapping interface rebinding in the configuration is shown in Listing
\ref{lst:interface_rebinding_recap}. Note that the only information the
configurator has for locating the replacement is the name of the interface and
its module.

\begin{listing}[H]
\begin{minted}{jolie}
interface Dummy = Concrete from "my-module"
\end{minted}

\caption{Rebinding the interface \txtl{Dummy} to match the interface
    \txtl{Concrete} from the module \txtl{my-module}}

\label{lst:interface_rebinding_recap}

\end{listing}

The interface will be found by parsing the module, starting at its entry point.
Its AST, if it exists, will then be searched for an interface matching the
replacement. Simply replacing the dummy with its replacement, however, isn't
enough. Consider, for example, the scenario shown in Listing
\ref{lst:interface_rebinding_types}. In this scenario, the processed AST will
no longer be valid due to the missing types (\txtl{FooRequest} and
\txtl{FooResponse}).

\begin{listing}[H]
\begin{minted}{jolie}
type FooRequest: void {
    .child: string
}

type FooResponse: int

interface Concrete {
    RequestResponse:
        foo(FooRequest)(FooResponse)
}
\end{minted}

\caption{Simply copying the interface definition is not enough, the types must
    also be copied}

\label{lst:interface_rebinding_types}
\end{listing}

The relevant type definitions are found by iterating through every operation of
the replacement interface. If the types listed in the operations are type links
those are followed. This leads to a recursive algorithm, which looks through
every type looking for type links, which will cause more types to be copied
over. In order to avoid infinite loops a set of already visited types is kept.
If a type has been seen before then its children will not be visited.

In the end, the original interface definition will be replaced, and all
necessary type definitions will also be added to the program.

\paragraph{Parameters}

Parameters being a new concept in Jolie simply outputs new nodes for every
single parameter assignment. This along with the definitions gathered from
ordinary Jolie code parsing are processed in the interpretation tree building
phase.

\subsection{Additions to the Semantic Verifier}

As we have seen, several syntax changes were also made to the core languages.
These modifications to the syntax changed which programs were
legal. As an example, interfaces are now allowed to be empty at a syntactical
level, to signal that they should be replaced by configuration. However having
an empty interface not being replaced should still cause an error.

This means that additional checking must now be performed, since it no longer
will be caught during parsing, to ensure that programs are semantically valid.
Such a phase already exists in the compiler, the semantic verifier.

The semantic verifier works by performing an AST traversal and validate the
semantics of each node.

The verifier has been updated to not allow empty interfaces, this was
previously enforced at a syntax level. Since the configurator runs before this
phase and updates the AST, this will only cause errors for non-configured
trees.

Non-dynamic ports are ensured to be constant by using features already
implemented by the semantic verifier.

The semantic verifier is the last phase of the Jolie interpreter when running
with the \txtl{--check} flag. As we will discuss in the coming section, it is
not practical to perform type-checking of parameters at this point. As a
result, it will not be checked using the \txtl{--check} flag.

\subsection{Additions to the Interpretation Tree Builder and Late Verification}

% Introduction

Interpretation tree building was expanded to support parameters. All other
features required no changes to AST structure, and could simply reuse existing
features.

This section will discuss the technical details of how the Interpretation Tree
Builder (ITB) works. It will cover issues that are specific to this particular
implementation of the Jolie engine.

% What is the ITB and what does it do?

The Interpretation Tree Builder (ITB) is the last step before a Jolie program
is ready to be executed. The ITB works by traversing the AST. The ITB will
collect information from the AST in the form of definitions and inform the AST
about their presence. These definitions include ports, interfaces, types, and
code procedures.

% Processes

Within code procedures the ITB will produce ``processes'' from the AST nodes
which are runnable. It is from processes the meat of code execution is
implemented.  For example, it is from processes everything from addition to
operation calls are implemented.

% When are types ready?

In this phase, the ITB will be gathering complete information about types. Type
building is, for the most part, a one-to-one translation from the AST nodes
into the internal representation. An important job of the ITB with respect to
type building is resolving of type references.  This turns references to a name
into an actual type reference.  Because types aren't built until this phase is
what makes type-checking during semantic verification problematic. The semantic
verifier would essentially have to duplicate the work of the ITB in order to
perform type-checking earlier.  This also means that type-checking cannot be
performed until the ITB is finished.

% State, and how the ITB changes initial state.

The ITB will need to initialize the system with certain values. For example,
the ITB will initialize the \joliel{location} and \joliel{protocol} variables
of output ports. These variables are local only to a single request. At this
point, however, the state of the Jolie program has not yet been initialized.

In Jolie, the state is associated with the current thread of execution. Jolie
uses a separate thread of execution for handling every message. As a result,
every single request gets its own fresh state. All state is copied from the
initialization execution thread. This thread performs work requested by the ITB
followed by the \joliel{init} procedure. Thus right after the ITB is done, and
the initialization execution thread is running, code execution starts. This
includes both user code and loading of any embedded services.

It should be noted that any process which works with state must be run on an
execution thread since it depends on the state being present on the thread.

% Handling of parameter definitions and assignments inside the ITB

The ITB receives two new node types from the configurator and code parser:
parameter assignments, and parameter definitions.

For parameter definitions, the ITB will resolve the types. Ensuring that all
links are properly resolved. This will output a ``processed'' parameter
definition which the ITB will inform the interpreter about.

In a similar fashion, the ITB will process parameter assignments and inform the
interpreter. Type-checking cannot be performed at this point, due to types not
being ready before the ITB finishes. Simply outputting processes for
configuration will make it impossible to perform type-checking before code
execution starts, due to their dependency on state.

The processed assignments contain an ordinarily processed LHS (of the
        assignment), which contains the variable path. The RHS has been
processed into an expression, which can be evaluated into a native Jolie value
on which type-checking can be performed.

% Introduction of new phase and working around lack of state

Because of these constraints a new ``late-checking phase'' has been created.
This phase runs right after the ITB, but before code execution starts. In this
phase, it is not possible to directly touch the init state. It is, however,
    possible to both create values, and evaluate expression. This phase creates
    a synthetic state by creating a new value which will act as the state.  The
    expressions are evaluated against this root and later type-checked. The
    result of evaluating the parameter assignments are then copied into the
    init state, assuming, of course, that type-checking was successful.

