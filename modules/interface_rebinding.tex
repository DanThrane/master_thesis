\section{Interface Rebinding}
\label{sec:interface_rebinding}

TODO Copy pasted from the paper, needs some introduction and re-wording (for
example it talks directly about circuit breakers)

Packaging proxy services like circuit breakers raises a particular problem.
Intuitively, the proxy should only accept calls for operations that are
declared in the interface of the target service. However, this interface is
known only at deployment time, since we may want to reuse the same circuit
breaker package to protect different services.
% In order to allow the creation of packable generic services, that use the
% aggregation and courier features, we have introduced parametric interfaces.
Proxy services are thus inherently \emph{parametric} on the interfaces of the
target services that we choose at deployment time. To address this problem, we
introduce the notion of configurable interface to Jolie.

\joliel{#ext interface ITarget}

In Listing \ref{lst:ext_interfaces} we define an externally configurable
interface. A concrete interface is bound to it at deployment time by reading
the configuration, giving the service, in this case \lstinline|CircuitBreaker|,
the information needed to correctly proxy operations. Observe that since we
want Jolie services to be type-checkable without knowing their configuration
(since configuration may change in different deployment setups), this means
that the behaviour of the service is necessarily defined as polymorphic, i.e.,
it cannot assume any specific operation in the configurable interfaces (this is
obtained through aggregation in Jolie, see~\cite{MGZ14}).

\subsection{Example: Writing a Generic Circuit BReaker}

TODO Copy pasted from the paper, expand on this.

% Jolie has a native construct for ``aggregating'' services, a generalisation
% of network proxies.

Proxy services delegate the computation of replies for their requests to other
services. A notable example is circuit breaker~\cite{N07}.
%
We summarise this pattern in the following
(see~\cite{DBLP:journals/corr/MontesiW16} for a thorough discussion in Jolie).

Circuit breakers attempt to protect against some of the problems that occur
when using remote calls, such as connection problems, timeouts, and critical
faults.
%
During normal operation, a \lstinline|CircuitBreaker| functions like a normal
proxy between a \lstinline|Client| and a \lstinline|TargetService|.  Monitoring
code inside of the \lstinline|CircuitBreaker| attempts to detect problems. If
enough problems are detected, the \lstinline|CircuitBreaker| will start failing
immediately without attempting to proxy the call. After a period of time it
will start allowing some calls through, and eventually transition back to the
normal state and allow all calls through.

To create a circuit breaker running in a client, we would embed the circuit
breaker locally and have it bound to our external payment processor, this is
shown in Listing \ref{lst:cb_col}.

\begin{listing}[H]
\begin{minted}{jolie}
profile "shop-production" configures "Shop" {
    outputPort PaymentProcessor embeds "cb-pp"
}
profile "cb-pp" configures "CircuitBreaker" {
    interface ITarget = PaymentProcessIface from "PaymentProcessor"
    outputPort TargetSrv { ... }
}
\end{minted}

\caption{cb.col: Shop configuration with client-side circuit breaker for
    \txtl{PaymentProcessor}}

\label{lst:cb_col}

\end{listing}

We can just as easily create a circuit breaker that operates server-side
(intercepting incoming calls), or as a proxy in the network, by adopting
different deployment files for \lstinline|"cb-pp"|.
