\section{Interface Rebinding}
\label{sec:interface_rebinding}

TODO Need to be more explicit about which features are new, and which are
existing features.

In this section we will first more closely inspect the existing aggregation,
   and courier features of Jolie. This leads to the introduction of interface
   rebinding which is explained following that.

Jolie provides the ``aggregation'' feature. We first introduced aggregation in
the background material for Jolie. In short the feature is a generalisation of
proxies. The aggregation feature simply forwards requests from one service to
another. A simple proxy to the calculator service is shown in Listing
\ref{lst:aggregation_example}.

\begin{listing}[H]
\begin{minted}{jolie}
include "calculator.iol" from "calculator"

inputPort Self {
    Location: "socket://localhost:12345"
    Protocol: sodep
    Aggregates: Calculator
}

outputPort Calculator {
    Location: "socket://calc.example.com:12345"
    Protocol: jsonrpc
    Interfaces: ICalculator
}
\end{minted}

\caption{A calculator proxy: This service will proxy any call to the calculator
    service bound in the output port \joliel{Calculator}}

\label{lst:aggregation_example}

\end{listing}

Aggregation can be extended by using couriers, which allows for the service to
run code associated with requests that are proxied. Couriers may work either
for an entire interface, or any particular operation. A courier may even
decide not to forward a particular call. Listing \ref{lst:simple_courier} shows
an extension for the calculator proxy with a courier.

\begin{listing}[H]
\begin{minted}{jolie}
courier Self { // The name of a courier matches its input port

    // Couriers can match an entire interface
    [interface ICalculator(request)(response) {
        println@Console("Received call for calculator!")();
        forward(request)(response)
    }]

    // Or just a particular operation
    [sum(request)(response)] {
        // The courier may choose to forward a request, or answer the
        // request itself
        if (#request.numbers > 2) { forward(request)(response) }
        else { response = request.numbers[0] + request.numbers[1] }
    }
}
\end{minted}

\caption{A courier allows additional code to run alongside a potential
    forwarding}

\label{lst:simple_courier}

\end{listing}

Additionally Jolie supports ``interface extenders''. These can, like the name
suggests, extend the types of operations (in interfaces) with additional
fields. These may add additional fields for every operation, or specific
operations. These can also add faults to the type signature of an operation.

Like with any other output port, the service relies entirely on the interface
listed in the output port to be correct. There is no communication with the
target service about the correct interface. If the interface doesn't match, it
will simply fail when attempting to invoke the operation.

\emph{A consequence of this is that it isn't possible to create entirely generic
proxies without knowing the interface.}

The aggregates and courier features allow for Jolie to implement many different
common proxy-like patterns. These patterns mostly deal with their target
service in a generic fashion, making very few, if any, assumptions about the
target service. However because the aggregations feature needs the interface it
would be impossible to write a fully generic proxy service.

The interface rebinding feature fixes this problem, by allowing configuration
files to redefine an interface at deployment time. This way the generic service
may write ordinary code, making no assumptions about the underlying service,
and only at deployment time it will know which interface the target service
has.

\begin{listing}[H]
\begin{minted}{jolie}
interface ITarget
\end{minted}
\caption{Configurable interfaces are defined by leaving the body empty}
\label{lst:ext_interfaces}
\end{listing}

The interface is bound from the configuration file, in a similar fashion to
other configurable units, as shown in Listing \ref{lst:reconf_interface}.

\begin{listing}[H]
\begin{minted}{jolie}
interface ITarget = ICalculator from "calculator"
\end{minted}

\caption{Rebinding the \joliel{ITarget} interface to the \joliel{ICalculator}
    interface from the calculator module}

\label{lst:reconf_interface}
\end{listing}


\subsection{Example: Writing a Generic Circuit Breaker}

TODO Copy pasted from the paper, expand on this.

% Jolie has a native construct for ``aggregating'' services, a generalisation
% of network proxies.

Proxy services delegate the computation of replies for their requests to other
services. A notable example is circuit breaker~\cite{N07}.
%
We summarise this pattern in the following
(see~\cite{DBLP:journals/corr/MontesiW16} for a thorough discussion in Jolie).

Circuit breakers attempt to protect against some of the problems that occur
when using remote calls, such as connection problems, timeouts, and critical
faults.
%
During normal operation, a \lstinline|CircuitBreaker| functions like a normal
proxy between a \lstinline|Client| and a \lstinline|TargetService|.  Monitoring
code inside of the \lstinline|CircuitBreaker| attempts to detect problems. If
enough problems are detected, the \lstinline|CircuitBreaker| will start failing
immediately without attempting to proxy the call. After a period of time it
will start allowing some calls through, and eventually transition back to the
normal state and allow all calls through.

To create a circuit breaker running in a client, we would embed the circuit
breaker locally and have it bound to our external payment processor, this is
shown in Listing \ref{lst:cb_col}.

\begin{listing}[H]
\begin{minted}{jolie}
profile "shop-production" configures "Shop" {
    outputPort PaymentProcessor embeds "cb-pp"
}
profile "cb-pp" configures "CircuitBreaker" {
    interface ITarget = PaymentProcessIface from "PaymentProcessor"
    outputPort TargetSrv { ... }
}
\end{minted}

\caption{cb.col: Shop configuration with client-side circuit breaker for
    \txtl{PaymentProcessor}}

\label{lst:cb_col}

\end{listing}

We can just as easily create a circuit breaker that operates server-side
(intercepting incoming calls), or as a proxy in the network, by adopting
different deployment files for \lstinline|"cb-pp"|.
