% Status: draft

\section{Introduction to Jolie}

We need to cover:

\begin{itemize}
\item Basic syntax
\item Control-flow, defines, interfaces, types, type-checking, that sort of
thing
\item Ports, protocols, that sort of thing
\item Basic example (steal this one from article)
\end{itemize}

% ---

Jolie is a service-oriented programming language, and is build to support a
microservice natively. In this section we will cover what kind of language
Jolie is, and how it is currently used.

Jolie has a C-inspired syntax, and is dynamically typed. Its interpreter is
written in Java.

The language has no native functions or methods, but instead works in
processes. A process has no arguments, and does not contain any stack (in the
case of recursive calls). There are two pre-defined processes, which will
always be called by the interpreter, these are called \verb!init! and
\verb!main!.

\begin{listing}[H]
\begin{minted}{java}
include "console.iol"

define PrintOutput {
    println@Console(output)() // Prints 'OK'
}

init { a = 1 }

main {
    b = 2;
    c = a + b; // c = 3
    if (c == 3) {
        output = "OK"
    } else {
        output = "Bad"
    };
    PrintOutput // Calls the defined process 'PrintOutput'
}
\end{minted}
\caption{A very simple Jolie program}
\label{lst:simple_jolie}
\end{listing}

Listing \ref{lst:simple_jolie} shows a very simple programming language, in
what looks like what you might expect from a dynamic language with C-inspired
syntax. However a few things may also strike you as odd.

First of all there are typos on lines 12, 14 or 16, the semicolon is not needed
here, in fact it would be a syntax error. The reason for this is that the
semicolon isn't used strictly for parsing purposes, but it instead for having
multiple statements in a process. The "semicolon" statement, also called a
sequence statement, has a syntax of \verb!A ; B!, which should be read as:
first perform statement \verb!A!, then perform \verb!B!. The sequence
statement requires both of the operands to be present, hence the syntax error.
Another similar statement is the parallel statement, which has a syntax of
\verb!A | B!, which reads as: do \verb!A! and \verb!B! in parallel. Using
these operators together allows the programmer to easily create a fork-join
workflow. This is typically used in microservices when we want to collect
data in parallel, and continue once all of the data has been retrieved.

Secondly has slightly different rules for scoping. In Jolie everything not
defined in the global scope goes into the same scope. This also persists
through calls to defines. This is the reason that \verb!PrintOutput! can use
the output variable.

Several execution modes exists. The default execution mode, which was used in
Listing \ref{lst:simple_jolie} is \verb!single!. This means that the
\verb!main!  process is run just a single time. Two more modes exists, those
being \verb!concurrent! and \verb!sequential!. TODO Some more stuff

Ports are the primitive that Jolie uses for communication, two types of ports
exists: input and output. Ports describe a running service, where it is located
(\verb!Location!), and how to speak to it (\verb!Protocol!), and finally which
operations it supports (\verb!Interfaces!)

Ports, interfaces, types

Embedded, aggregates, redirection
