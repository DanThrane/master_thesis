\section{Appendix A: JPM Manifest Specification}\label{package-specification}

This document covers the specification of the file which defines a
package. The format used for this document will be JSON, but the format
and whether or not to allow for several documents is still up for
discussion. For now we should avoid using any features which the generic
Jolie value cannot support.

\subsection{Purpose}\label{purpose}

The purpose of the package document is to define what a package is.
Every Jolie package will contain such a document, and it describes
several important properties about the package. These properties are
described in the section ``Format and Properties''.

\subsection{Table of Contents}\label{table-of-contents}

\begin{itemize}
\tightlist
\item
  \protect\hyperlink{format-and-properties}{Format and Properties}

  \begin{itemize}
  \tightlist
  \item
    \protect\hyperlink{name}{name}
  \item
    \protect\hyperlink{version}{version}
  \item
    \protect\hyperlink{license}{license}
  \item
    \protect\hyperlink{authors}{authors}
  \item
    \protect\hyperlink{private}{private}
  \item
    \protect\hyperlink{main}{main}
  \item
    \protect\hyperlink{dependencies}{dependencies}
  \item
    \protect\hyperlink{dependency}{dependency}

    \begin{itemize}
    \tightlist
    \item
      \protect\hyperlink{name-1}{name}
    \item
      \protect\hyperlink{version-1}{version}
    \item
      \protect\hyperlink{registry}{registry}
    \end{itemize}
  \item
    \protect\hyperlink{registries}{registries}
  \item
    \protect\hyperlink{registry-1}{registry}

    \begin{itemize}
    \tightlist
    \item
      \protect\hyperlink{name-2}{name}
    \item
      \protect\hyperlink{location}{location}
    \end{itemize}
  \end{itemize}
\end{itemize}

\hypertarget{format-and-properties}{\subsection{Format and
Properties}\label{format-and-properties}}

\hypertarget{name}{\subsubsection{name}\label{name}}

\textbf{Name:} \texttt{name}

\textbf{Optional:} false

\textbf{Type:} \texttt{string}

\textbf{Description:} The \texttt{name} property uniquely defines a
package in a registry. Every registry must only contain a single package
with a given name.

\textbf{Rules:}

\begin{itemize}
\tightlist
\item
  The name of a package is \emph{not} case-sensitive
\item
  The length of a name is less than 255 characters
\item
  Names are US-ASCII
\item
  Names may only contain unreserved URI characters (see section 2.3 of
  \href{https://www.ietf.org/rfc/rfc3986.txt}{RFC 3986})
\end{itemize}

If any of these rules are broken the JPM tool should complain when
\emph{any} command is invoked. Similarly a registry should reject any
such package.

\hypertarget{version}{\subsubsection{version}\label{version}}

\textbf{Name:} \texttt{version}

\textbf{Optional:} false

\textbf{Type:} \texttt{string}

\textbf{Description:} This property describes the current version of
this package.

\textbf{Rules:}

\begin{itemize}
\tightlist
\item
  The version string must be a valid SemVer 2.0.0 string (see
  http://semver.org/spec/v2.0.0.html)
\end{itemize}

\hypertarget{license}{\subsubsection{license}\label{license}}

\textbf{Name:} \texttt{property\_name}

\textbf{Optional:} false

\textbf{Type:} \texttt{string}

\textbf{Description:} Describes the license that this package is under.

\textbf{Rules:}

\begin{itemize}
\tightlist
\item
  Must be a valid identifier. See https://spdx.org/licenses/
\end{itemize}

\hypertarget{authors}{\subsubsection{authors}\label{authors}}

\textbf{Name:} \texttt{authors}

\textbf{Optional:} false

\textbf{Type:}
\texttt{string\textbar{}array\textless{}string\textgreater{}}

\textbf{Description:} Describes the authors of this package

\textbf{Rules:}

\begin{itemize}
\tightlist
\item
  The array must contain at least a single entry
\item
  Each entry should follow this grammar:
\end{itemize}

\begin{verbatim}
name ["<" email ">"] ["(" homepage ")"]
\end{verbatim}

\hypertarget{private}{\subsubsection{private}\label{private}}

\textbf{Name:} \texttt{private}

\textbf{Optional:} true

\textbf{Type:} \texttt{boolean}

\textbf{Description:} Describes if this package should be considered
private. If a package is private it cannot be published to the
``public'' repository.

\textbf{Rules:}

\begin{itemize}
\tightlist
\item
  By default this property has the value of \texttt{true} to avoid
  accidential publishing of private packages.
\end{itemize}

\hypertarget{main}{\subsubsection{main}\label{main}}

\textbf{Name:} \texttt{main}

\textbf{Optional:} true

\textbf{Type:} \texttt{string}

\textbf{Description:} Describes the main file of a package.

\textbf{Rules:}

\begin{itemize}
\tightlist
\item
  The value is considered to be a relative file path from the package
  root.
\end{itemize}

\hypertarget{dependencies}{\subsubsection{dependencies}\label{dependencies}}

\textbf{Name:} \texttt{dependencies}

\textbf{Optional:} true

\textbf{Type:} \texttt{array\textless{}dependency\textgreater{}}

\textbf{Description:} Contains an array of dependencies. See the
``dependency'' sub-section for more details.

\textbf{Rules:}

\begin{itemize}
\tightlist
\item
  If the property is not listed, a default value of an empty array
  should be used
\end{itemize}

\hypertarget{dependency}{\subsubsection{dependency}\label{dependency}}

\textbf{Type:} \texttt{object}

\textbf{Description:} A dependency describes a single dependency of a
package. This points to a package at a specific point on a specific
registry.

\hypertarget{name-1}{\paragraph{name}\label{name-1}}

\textbf{Name:} \texttt{name}

\textbf{Optional:} false

\textbf{Type:} \texttt{string}

\textbf{Description:} Describes the name of the dependency. This refers
to the package name, as defined earlier.

\textbf{Rules:} A dependency name follows the exact same rules as a
package name.

\hypertarget{version-1}{\paragraph{version}\label{version-1}}

\textbf{Name:} \texttt{version}

\textbf{Optional:} false

\textbf{Type:} \texttt{string}

\textbf{Description:} Describes the version to use

\textbf{Rules:}

\begin{itemize}
\tightlist
\item
  Must be a valid SemVer 2.0.0 string
\item
  (This property follows the same rules as the package version does)
\end{itemize}

\hypertarget{registry}{\paragraph{registry}\label{registry}}

\textbf{Name:} \texttt{registry}

\textbf{Optional:} true

\textbf{Type:} \texttt{string}

\textbf{Description:} This describes the exact registry to use. If no
registry is listed the ``public'' registry will be used.

\textbf{Rules:}

\begin{itemize}
\tightlist
\item
  The value of this property must be a valid registry as listed in the
  \texttt{registries} property.
\end{itemize}

\hypertarget{registries}{\subsubsection{registries}\label{registries}}

\textbf{Name:} \texttt{registries}

\textbf{Optional:} true

\textbf{Type:} \texttt{array\textless{}registry\textgreater{}}

\textbf{Description:} Contains an array of known registries. See the
registry sub-section for more details.

\textbf{Rules:}

\begin{itemize}
\tightlist
\item
  This property contains an implicit entry which points to the public
  registry. This registry is named ``public''.
\end{itemize}

\hypertarget{registry-1}{\subsubsection{registry}\label{registry-1}}

\textbf{Type:} \texttt{object}

\textbf{Description:} A registry describes a single JPM registry. A JPM
registry is where the package manager can locate a package, and also
request a specific version of a package.

\hypertarget{name-2}{\paragraph{name}\label{name-2}}

\textbf{Name:} \texttt{name}

\textbf{Optional:} false

\textbf{Type:} \texttt{string}

\textbf{Description:} This property uniquely identifies the registry.

\textbf{Rules:}

\begin{itemize}
\tightlist
\item
  A name cannot be longer than 1024 characters
\item
  The name cannot be ``public''
\item
  No two registries may have the same name
\end{itemize}

\textbf{TODO:}

\begin{itemize}
\tightlist
\item
  Encoding of name
\item
  Should the length limit be dropped? There is no technical reason for
  the limit
\end{itemize}

\hypertarget{location}{\paragraph{location}\label{location}}

\textbf{Name:} \texttt{location}

\textbf{Optional:} false

\textbf{Type:} \texttt{string}

\textbf{Description:} Describes the location of the registry.

\textbf{Rules:}

\begin{itemize}
\tightlist
\item
  Must be a valid Jolie location string (e.g.
  ``socket://localhost:8080'')
\end{itemize}
