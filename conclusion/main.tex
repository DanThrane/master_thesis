In this thesis, we have presented two new extensions to the Jolie programming
language, namely, the module system and the configuration system.

The module system allows the Jolie engine to think about collections of files
which are rooted in a particular directory. This introduced the new module
include primitive.

The configuration system is built upon the module system to allow configuration
of modules. The configuration system introduced the new configuration format
(COL) which allows the user configuration of modules. COL files entirely
replace the old way of doing configuration, through ordinary Jolie source
files. COL files present a way of configuring a module, in a similar syntax to
that of ordinary Jolie code.

Input ports and output ports, the primitives used by Jolie for communication,
can be configured. The configuration of ports, follow the features that
the Jolie language already provided. As a result, it is now possible to
move freely between externally bound output ports to embedding the same
service, entirely from the configuration. This was previously not possible
without changing the source code of a service.

Parameters allow for configuration values to be provided to a service.
Parameters are automatically type-checked at deployment time of a service. Due
to constraints within the Jolie engine, a new verification stage was added to
the Jolie engine.

The configuration system introduced a notion of interface parametricity through
interface rebinding. This feature allows developers to write generic Jolie
services which use the, already existing, aggregation feature. This was also
not previously doable without changing the source code of the service.

We also presented the Jolie Package Manager (JPM). JPM manages a new concept of
packages. Packages are an extension of the Jolie modules, also developed in
this thesis. Package extends modules, by adding a package manifest which
describes the package.

Packages can be published to a registry. Published packages are downloaded from
the registry, by the JPM tool, and installed into packages.  The JPM tool also
provides a variety of features that include registry account management,
lifetime hooks, and integrity checking of packages.

