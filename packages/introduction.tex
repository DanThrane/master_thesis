\section{Introduction}

A Jolie Package is an extension of a Jolie Module. Recall that a Jolie Module
was defined as a directory root, a name, and optionally an entry-point
for the module. A package extends this concept by adding information required
for package management.

A Jolie Package is described by a package manifest. The package manifest is a
JSON file, which is always placed at the root of the package, and must be
called \verb!package.json!. The fixed location allows for the package manager
to easily identify a package. The JSON format was chosen as it plain-text, and
easy to read and write for both humans and machines.

In listing \ref{lst:simple_manifest} we show a simple package manifest. This
manifest showcases the most important features of the manifest. A complete
specification of the package manifest format can be seen in Appendix 1. The
service that this manifest describes is shown in Figure \ref{fig:simple_calc}.

\begin{listing}[H]
\begin{minted}{json}
{
    "name": "calculator",
    "main": "main.ol",
    "description": "A simple calculator service",
    "authors": ["Dan Sebastian Thrane <dathr12@student.sdu.dk>"],
    "license" "MIT",
    "version": "1.0.0",
    "dependencies": [
        { "name": "addition", "version": "1.2.X" },
        { "name": "multiplication", "version": "2.1.0" }
        { "name": "numbers", "version": "1.0.0" }
    ]
}
\end{minted}
\caption{A simple package manifest}
\label{lst:simple_manifest}
\end{listing}

This manifest configures the ongoing example of the calculator system.  Lines
2-3 take care of the module definition, the root is implicitly decided by the
location of the manifest. The remaining attributes, however, are entirely
unique to packages. Some attributes included in the manifest are there for
indexing and discoverability purposes, examples of such attributes are shown in
lines 4-6. The version we will cover shortly, but it is used for package
management. The rest of the manifest describes the dependencies of this
package.


