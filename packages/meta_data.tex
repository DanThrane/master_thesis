\section{Package Manifest}

In this section we'll cover, in details, the format of Jolie manifests. The
complete specification for the package manifest can be found in Appendix 1.
However in this section we'll also cover some of the reasoning behind the
choices found in Appendix 1.

\subsection{The {\tt name} Attribute}

The \mintinline{text}{name} attribute uniquely identifies a package. We'll
start this discussion by looking at the constraints imposed on it, from
Appendix 1:

\begin{enumerate}
\item
    The name of a package is \emph{not} case-sensitive
\item
    The length of a name is less than 255 characters
\item
    Names may only contain unreserved URI characters (Section 2.3 of
            \autocite{rfc3986})
\item
    Names are US-ASCII
\item
    Every registry must only contains a single package with a given name.
\end{enumerate}

Points 1 through 4 all contribute towards a common goal: packages should be
displayable in URLs. This will allow us to more easily create, for example,
a web application where a developer can browse through packages.

Putting some constraints on the package name is also helpful in multiple places
of the package manager. Consider for example a system that might store packages
organized by their name (say for example a caching mechanism). If one package
would be named ``../foo'', then that might accidentally override the contents
of the package named ``foo''. Of course these name constraints are not a
general solution for these types of injection attacks, but it does allow for
several services working with packages to not worry as badly about the package
names.

It should also be noted that the package names are not compatible with the
rules of identifiers. This has affected the design of the Jolie module system.
This is, for example, the reason that strings are used in both configuration
files and module includes, when making references to other modules.

Rule 4 states that package names are unique within a single registry. But this
constraint actually goes even further, since these packages directly extend
from the modules, we cannot have a system containing two packages named the
same thing, even if they reside in two different registries. Generally we make
no assumptions that registries communicate with each other, for this reason it
is perfectly allowed that two distinct registries may contain name clashes, but
any single package must not have a dependency which clashes either with another
dependency, or the package who has the dependency itself.

\subsection{The Meta Data Attributes ({\tt license, authors, description})}

These attributes are mostly used for search results. They allow for a developer
seeking information about a package, to quickly gather the most important
information.

License identifiers are validated against a pre-existing list of license
identifiers. This is done to ensure that typos are not made in license names.
The identifiers used are from the SPDX license list, which contains a list of
commonly found licenses. Currently only the license identifiers are supported,
this could be extended to use SPDX License Expression Syntax, which allows for
references to custom licenses.

\subsection{The {\tt version} Attribute}

% source: Semantic Versioning 2.0.0 - Tom Preston-Werner - http://semver.org/

The \mintinline{text}{version} attribute used by packages declare which version
of the source code the manifest describes. This field uses the Semantic
Versioning (SemVer) specification for version numbers. It is a requirement to
use SemVer versions, if a version is specified, which is not a valid SemVer
version, then validation of the manifest will fail. Enforcing this standard
allows for several other features, which will be covered in a moment.

SemVer version numbers consists of three fields, all integers, those being the
\emph{major}, \emph{minor}, and \emph{patch} version numbers. Additionally a
version may have a pre-release label or additional meta data. Labels and build
meta-data won't be covered, as they are not that important for our use-case.
The (simplified) syntax of SemVer version number is shown in Listing
\ref{lst:semver_syntax}.

\begin{listing}[H]
\begin{minted}{text}
<major>.<minor>.<patch>[-<label>][+<meta-data>]
\end{minted}
\caption{Simplified syntax of a SemVer version number}
\label{lst:semver_syntax}
\end{listing}

Precedence of version numbers are determined by comparing each unit numerically
in the order of major, minor, and finally patch. The first differentiating
units determine the precedence. Thus, for example: 1.0.0 $<$ 1.0.1 $<$ 1.1.0
$<$ 1.10.0 $<$ 2.0.0.

Semantic Versioning dictates rules for how version numbers should change,
depending on how the public API changes. The rules can be summarized as:

\begin{enumerate}

    \item The patch version number \emph{must} be incremented, if only
        backwards compatible bug fixes are introduced.

    \item The minor version number \emph{must} be incremented, if backwards
        compatible functionality is introduced.

    \item The major version number \emph{must} be incremented, if incompatible
        changes are introduced.

\end{enumerate}

Having a major version of 0, indicates that the product is not yet stable. At
this point normal rules for incrementing version numbers do not apply. As a
result version 1.0.0 indicates the first public API.

Semantic Versioning has become a commonly used standard across many different
package managers, examples include
Cargo\footnote{\url{http://doc.crates.io/manifest.html}} and
NPM\footnote{\url{https://docs.npmjs.com/getting-started/using-a-package.json}}.

TODO Highlight benefits of using semantic versioning.

% source: https://docs.npmjs.com/misc/semver

For dependencies the version field allows for a bit more flexibility using
SemVer expressions, also known as SemVer ranges. These expressions allows the
developer more freedom in which version to use. In short the expressions allows
the developer to both whitelist certain ranges for use, and blacklist certain
ranges for use.

This provides both a convenience factor, i.e. the developer can express that
she doesn't care which specific version is used, as long as it has feature X
(released in some known version).

This feature also provides benefits for when the dependency tree is to be
calculated. Consider, for example, two dependencies A and B which both depend
on C with the same level of functionality, which was released in version 1.2.0.
Assuming that expressions were not allowed, these two services would have to
describe their dependency in an exact version. Service A might have been
written at a time when 1.2.3 of service C was the newest, while B was written a
bit later and is thus depending on version 1.2.6. The problem is now that the
client depending on A, and B, cannot do so, because A and B have conflicting
requirements, despite the fact that they both require the same level
functionality. With SemVer expressions these services may more accurately
describe what they actually depend on, service A might state that it depends on
">=1.2.0", while B is a bit more conservative and wants exactly version 1.2.6.
In this case the package manager will be able to choose version 1.2.6, since
this version fulfills the requirements of all dependencies.

Allowing for SemVer expressions to be used, is however not without its
drawbacks. In Section \ref{sec:lockfiles} we cover one of these problems, and
its solution.

\subsection{The Lifetime Attributes ({\tt events}, {\tt main})}

These attributes control aspects of the package related to the execution of
scripts, by the package manager, related to the package.

The \mintinline{text}{main} attribute controls the entry-point of the module.
This is used both in informing the Jolie Engine where the entry-point is
located. The attribute is also used for starting the package, we discuss this
in Section TODO.

The \mintinline{text}{events} attribute defines a series of lifetime hooks,
which are scripts that are run when specific lifetime events occur, this is
described in Section \ref{sec:lifetime_hooks}.

\subsection{The {\tt registries} Attribute}

The package manager may communicate with one or more registries, when
performing its work. This may either be to install dependencies, or it could be
to perform some of the more organizational commands, such as creating an
account.

For this the package manager will need to know about the different registries.
This is what the \mintinline{text}{registries} attribute describes, it is an
array containing definitions for all known registries (known to the package).

A registry is defined by a unique (within the package) name, and a location (a
URI following the same rules that Jolie does for its port locations). Listing
\ref{lst:registry_definition} shows an example.

\begin{listing}[H]

\begin{minted}{json}
{
    "name": "my-registry",
    "location": "socket://registry.example.com:9999"
}
\end{minted}

\caption{A registry named \mintinline{text}{my-registry} being hosted at
    \mintinline{text}{registry.example.com} running on port 9999}

\label{lst:registry_definition}

\end{listing}

Every package manifest has an implicit entry which is named ``public''. This
entry points to the public registry. Every command which needs to use a
registry, will optionally receive a registry name. If this name is specified,
it will look within the \mintinline{text}{registries} attribute. If no name is
specified, the ``public'' entry will be used.

TODO Should we talk about why it is good to have multiple registries here?
