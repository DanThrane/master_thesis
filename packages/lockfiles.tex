\section{Lock Files}
\label{sec:lockfiles}

% https://www.martinfowler.com/articles/continuousIntegration.html
% https://en.wikipedia.org/wiki/Continuous_integration#cite_note-2

Lock files are a feature designed to solve some of the drawbacks associated
with allowing SemVer expressions in the version field of dependencies. To see
how SemVer expressions pose a problem, we must first inspect the typical
engineering process in which it is used. This feature is by no means original,
prior work includes: Cargo\autocite{CRAA}, and Yarn\autocite{YARNB}.

Continuous integration (CI) is a software development process\autocite{MFCI}.
This process is used to avoid the so called ``integration hell'', in which the
time taken to integrate, or merge, the changes from multiple developers into a
single track, takes longer than developing the individual features.  The
process works by, multiple times a day, integrating all these changes, this way
failure in integration is found much earlier. We'll quickly summarize one
possible implementation of CI here. This is not the only way to do it, but
summarizes the most important parts, which we'll use to highlight potential
problems with SemVer expressions.

For this process to work efficiently it frequently uses automated tests and
builds, in order to perform this integration. The automated tests and builds
are then performed by a dedicated server. The server will most likely pull the
source code from a single repository, which represents the most up-to-date
track of development.

When new features are developed, the developer will pull the most up-to-date
source code, and develop the feature locally with this. This includes
performing all of the builds and tests. Thus when the feature is pushed back
onto the source control all of these tests should still pass. The role of the
CI server is then to verify that the process is upheld.

How the build and testing is performed obviously depend on which technologies
are used. A typical JPM based project might perform the following steps:

\begin{enumerate}
\item Pull source code from version control
\item Perform the build
    \begin{enumerate}
        \item Download JPM dependencies (i.e. \verb!jpm install!)
        \item Potentially build other necessary artifacts (such as Java
                libraries)
    \end{enumerate}
\item Run tests
\end{enumerate}

It is in step 2(a) the need for lock files arises, the problem is most easily
demonstrated through an example.

Consider the following scenario: a developer wishes to develop a new feature
which uses package X of at least version 2.1.0. For this reason the developer
adds the following line to the dependencies section: \mintinline{json}{{
"name": "X", "version": ">=2.1.0" }}. Afterwards the developer performs an
install of this package, at the moment this dependency to version 2.1.3. At
this point the developer continuous with developing the feature, without
having to perform any additional installs of the package. Once the feature
is developed, the developer performs all of the tests, and see that they
pass.

At this point the code is pushed onto the source control system, where it is
picked up by the CI server for testing. The CI server will then perform all of
the steps listed above, including installing the dependencies anew. However,
since the initial install by the developer had been performed a new version
has been released, say version 2.3.0. This version still fulfills the
expression listed in the dependency, and is hence chosen, since this is the
best fitting package for this dependency. Remember, the CI server has no way
of knowing which package was originally installed, it only has the package
manifest to work with.

Installing a different version of the package may however lead to the tests no
longer working! As a result, we cannot guarantee what works on a developer's
machine, necessarily works on any other machines that might try to run it.
As a result, some might choose to never use SemVer expressions to avoid this
pitfall.

Lock files offer a compromise between these two extremes. A lock file contains
the exact versions that every dependency was resolved to. The lock file should
be put into source control, thus when another machine attempts to perform the
build the package manager will know which exact version should be used for the
build.

The convenience factor of SemVer expressions is present. When we perform the
initial install, we might not care so which version to use, and the package
manager will still be able to just pick the best version for us.  Additionally
the package manager can provide upgrade scripts, which uses these expressions
as guidelines.

The lock files is placed in the package directory, and is called
\verb!jpm_lock.json!. Listing \ref{lst:lock_file} shows an example of a lock
file, which would have been generated for the scenario above.

The lock files of a dependency is \emph{not} used by a client package. As a
result library developers should still be careful not specifying too wide
expressions. This feature is only intended to guarantee that a package will
have the same dependencies regardless of which machine it runs on. It is not
intended as a replacement for specifying exact versions. If a package requires
a specific version, then that should be reflected in the package manifest.

\begin{listing}[H]
\begin{minted}{json}
{
    "_note": "Auto-generated file notice",
    "locked": {
        "X@>=2.1.0/RegistryName": {
            "resolved": "2.1.3",
            "checksum": "..."
        }
    }
}
\end{minted}

\caption{A lock file showing that the dependency for package X of at least
    version 2.1.0 has been resolved to version 2.1.3}

\label{lst:lock_file}
\end{listing}
