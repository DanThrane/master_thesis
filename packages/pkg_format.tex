\section{The {\tt .pkg} Format} \label{sec:pkg}

The \verb!.pkg! file format, is the format that JPM uses for distribution of
packages. Simply the file format contains a Jolie package zipped up into a
single file.

Currently the contents of the zip file maps directly with the underlying
package. The format is however open to extension, by allowing for new files to
be introduced that can be used in various extensions. One such extension could
for example be native code signing, which we discuss in more details in Section
TODO. An obvious implementation strategy would involve adding the necesary
files directly into the ZIP archive.

Extending the ZIP file format is a quite common occourence. Espescially when
shipping what is essentially a collection of files. Popular examples of other
file formats using this approach includes the JAR\footnote{Package file format
    typically used for aggregating many Java class files together} format, and
    Office Open XML\footnote{Developed by Microsoft, used in their office
        applications}.
