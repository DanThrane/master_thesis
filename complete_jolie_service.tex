\section{A Complete Application with Jolie}

In this section we will describe a complete application written in Jolie. The
application will contain several services, and will be written using best
practices from before the module and package system.

\subsection{Architecture}

% TODO Things we wish to cover in this section:
%
%   - Includes and why they will be problematic
%     + Can be resolved by using another service's interface and types
%   - Configuring location of ports, both external and embedding of stuff
%     + Show this by having a service be external, and another embedded
%   - Project structure, and why this is a problem
%     + Show the complete structure of a single service, with several deps
%   - Parameter configuration
%     + Show this by configuring a database

The application we will be building is a very simple shop application. This
\emph{shop} application will be able to look in inventory, and reserve a
product, and arrange for \emph{shipping}, while charging \emph{payment} from
the customer.

The architecture of our application is shown in Figure \ref{fig:complete_arch}
we see an illustration of the system's architecture. The \emph{shop} service
will be contacted by the front-end service, which in this case is
\emph{web-store}.

The dashed region (TODO) displays services that should be running in the same
Jolie engine. This is accomplished via embedding, which was introduced in a
previous section.

\begin{listing}[H]
\begin{minted}{text}
+---------------+     +---------------+     +---------------+
| web-store     | --> | shop          | --> | payment       |
+---------------+     +---------------+     +---------------+
                              |        \
                              |         \    +---------------+
                              v          --> | shipping      |
                      +---------------+      +---------------+
                      | inventory     |
                      +---------------+
\end{minted}
\caption{The Architecture of a Simple Microservice System}
\label{fig:complete_arch}
\end{listing}

\subsection{Implementing the Shop Service}

We will keep our focus on the \emph{shop} service, and it's interacting with
peering services. Putting together the stuff learned from the previous
sections, we can quickly setup an input port for the service, which has the
appropriate interface. It is considered best practice to place the public
interfaces that a service exposes in its own separate file. Files intended for
other services to include typically have the file extension \verb!.iol! as
opposed to \verb!.ol!. There is no technical difference between the two, but it
allows for the developer to more easily express intent.

Thus in order to implement our shop service we create two files, one for the
service implementation (\verb!shop.ol!), and another which can be used by other
services (\verb!shop.iol!).

\begin{listing}[H]
\begin{minted}{java}
// shop.iol
interface IShop {
    RequestResponse:
        checkout(CheckoutReq)(CheckoutRes),
        browse(BrowseReq)(BrowseRes)
}

// shop.ol
include "console.iol"
main {
    [checkout(request)(response) {
        println@Console("Implementation goes here")()
    }]

    [browse(request)(response) {
        println@Console("Implementation goes here")()
    }]
}
\end{minted}
\caption{TODO Caption}
\label{lst:simple_start}
\end{listing}

The operations that the shop exposes, needs to collaborate with the shipping,
payment, and inventory services. In order for us to speak to them they need
output ports.

First of all the output ports needs interfaces. Like we did with the shop
service, the other services have exposed their interfaces in a special file
intended for inclusion. As a result we will have to copy the
\mintinline{text}{.iol} files of these services into our own.

Secondly these output port needs to be reached. We can either embed the
services, making it run inside of the same Jolie engine as our shop service,
our we can provide external bindings to it. For output ports we may change this
binding dynamically at runtime. Note that this is unlike input ports which must
be ready at deployment time.

Binding an output port to an external service is relatively easy. For example
to let the \verb!Payment! port bind to a service using https, we might write
\mintinline{java}{Location: "socket://paymentprocessor.com:443"} and
\mintinline{java}{Protocol: https}. The input port at the payment processor
would also have to match this, to ensure it runs on the correct port and speaks
the correct protocol.

It is a similar to bind an output port to an embedded service. However this is
done by setting the \mintinline{java}{Location} or \mintinline{java}{Protocol}
attributes. We must instead instruct the engine to embed the service, which
most importantly requires us to point to some executable service. The Jolie
engine supports several language for these embedded services, including Jolie,
Java, and JavaScript. For our desired deployment, we wanted to embed the
inventory service inside of the shop service. Assuming that the inventory
service is written as a Jolie service, and its service is implemented in
\mintinline{text}{inventory.ol}, then we may create an embedding as shown in
Listing \ref{lst:simple_embedding}. Just like in the case of the external
services, the input port of the receiving service \emph{must match}. In the
case of embedded service there must be an input port listining on the
\mintinline{java}{"local"} location.

Quite often the location of an input port is considered a deployment problem.
We see this quite clearly in the case of embedding a service. All current
solutions in Jolie, require us to \emph{modify the source code of a service},
simply to change where the service should listen. Best practices in Jolie
attempt to make this less of a problem by including a configuration file which
contains constants. The inventory service, might include a file called
\verb!inventory_config.iol! with constants setting up the location and protocol
of the service. An example of this is shown in Listing
\ref{lst:include_as_conf}.

\begin{listing}[H]
\begin{minted}{java}
// inventory_config.iol
constants {
    INVENTORY_LOCATION = "local"
    INVENTORY_PROTOCOL = sodep
}

// inventory.ol
include "inventory_config.iol"

inputPort Inventory {
    Location: INVENTORY_LOCATION
    Protocol: INVENTORY_PROTOCOL
}
\end{minted}

\caption{A common Jolie practice for solving configuration of a service, is to
    include a file containing constants with the desired configuration.}

\label{lst:include_as_conf}

\end{listing}

\begin{listing}[H]
\begin{minted}{java}
embedded {
    Jolie:
        "inventory.ol" in Inventory
}
\end{minted}

\caption{Embedding the \mintinline{java}{inventory} service in the
    \mintinline{java}{Inventory} output port}

\label{lst:simple_embedding}

\end{listing}

With the code from Listing \ref{lst:simple_start} where the output port
\mintinline{java}{Console} is defined. The output port points to an embedding
of the console service, and is included directly in the \verb!console.iol!
file. This is a fairly common pattern used in Jolie, especially for services
that work in a library-like fashion (i.e. not intended as a stand-alone
service). This pattern is used for almost every single service in the Jolie
standard library.

With the output ports correctly configured, we may now implement the actual
business logic for our shop. For completeness sake this might look like shown
in Listing \ref{lst:op_impl}.

\begin{listing}[H]
\begin{minted}{java}
[checkout(request)(response) {
    // Do some local calculations
    checkForStock@Inventory(/* ... */)(hasStock);
    if (!hasStock) throw(OutOfStockFault);
    reserve@Inventory(/* ... */);
    charge@Payment(/* ... */)();
    send@Shipping(/* ... */)()
}]
\end{minted}
\caption{Implementing the checkout operation}
\label{lst:op_impl}
\end{listing}

\begin{listing}[H]
\begin{minted}{text}
.
+-- include
|   +-- shop.iol
|-- inventory.iol
|-- inventory.ol
|-- payment.iol
|-- shipping.iol
+-- shop.ol
\end{minted}
\caption{File Structure of the Shop Service}
\label{fig:file_structure}
\end{listing}

