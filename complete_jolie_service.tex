\section{A Complete Application with Jolie}

In this section we will describe a complete application written in Jolie. The
application will contain several services, and will be written using best
practices from before the module and package system.

Along the way several patterns used for re-using code will be highlighted. This
will serve as the core motivation for our solution.

\subsection{Architecture}

% TODO Things we wish to cover in this section:
%
%   - Includes and why they will be problematic
%     + Can be resolved by using another service's interface and types
%   - Configuring location of ports, both external and embedding of stuff
%     + Show this by having a service be external, and another embedded
%   - Project structure, and why this is a problem
%     + Show the complete structure of a single service, with several deps
%   - Parameter configuration
%     + Show this by configuring a database

The application we will be building is a very simple calculator system. The
architecture of our application is shown in Figure \ref{fig:complete_arch} we
see an illustration of the system's architecture.

The calculator service will provide various operations for applying an operator
on a sequence of numbers. In order to perform this, highly complex, action of
applying these operators, the calculator will contact other microservices.

The dashed region (TODO) displays services that should be running in the same
Jolie engine. This is accomplished via embedding, which was introduced in a
previous section.

\begin{listing}[H]
\begin{minted}{text}
+---------------+     +---------------+     +---------------------+
| web-calc      | --> | calculator    | --> | multiplication      |
+---------------+     +---------------+     +---------------------+
                              |
                              |
                              v
                      +---------------+
                      | addition      |
                      +---------------+
\end{minted}
\caption{The Architecture of a Simple Microservice System}
\label{fig:complete_arch}
\end{listing}

\subsection{Implementing the Calculator Service}

We will keep our focus on the \emph{calculator} service, and its interacting
with other services. Putting together the stuff learned from the previous
sections, we can quickly setup an input port for the service, which has the
appropriate interface. It is considered best practice to place the public
interfaces that a service exposes in its own separate file. Files intended for
other services to include typically have the file extension \verb!.iol! as
opposed to \verb!.ol!. There is no technical difference between the two, but it
allows for the developer to more easily express intent. Files that are included
can be placed in the \verb!include! directory, this directory is always
implicitly added to the search path.

Thus in order to implement our calculator service we create two files, one for
the service implementation (\verb!calculator.ol!), and another which can be
used by other services (\verb!calculator.iol!). The implementation and
interfaces are shown in Listing \ref{lst:simple_start}.

\begin{listing}[H]
\begin{multicols}{2}

\begin{minted}[breaklines,fontsize=\footnotesize]{java}
// calculator.ol
include "console.iol"

inputPort Calculator {
    Location: "socket://localhost:12345"
    Protocol: sodep
    Interfaces: ICalculator
}

main {
    [sum(request)(response) {
        println@Console("Implementation goes here")()
    }]

    [product(request)(response) {
        println@Console("Implementation goes here")()
    }]
}
\end{minted}

\columnbreak

\begin{minted}[breaklines,fontsize=\footnotesize]{java}
// calculator.iol
type Numbers: void {
    .numbers[2, *]: int
}

interface ICalculator {
    RequestResponse:
        sum(Numbers)(int),
        product(Numbers)(int)
}
\end{minted}
\end{multicols}
\caption{TODO Caption}
\label{lst:simple_start}
\end{listing}

\begin{observation}
    TODO Keep interfaces and types in their own file, to make it easier to
    import.
\end{observation}

The operations that the calculator exposes, needs to collaborate with the other
services. In order for us to speak to them they need output ports.

First of all the output ports needs interfaces. Like we did with the calculator
service, the other services have exposed their interfaces in a special file
intended for inclusion. As a result we will have to copy the
\mintinline{text}{.iol} files of these services into our own.

Secondly these output port needs to be reached. We can either embed the
services, making it run inside of the same Jolie engine as our calculator
service, or we can provide external bindings to it. For output ports we may
change this binding dynamically at runtime. Note that this is unlike input
ports which must be ready at deployment time.

Binding an output port to an external service is relatively easy. For example
to let the \verb!multiplication! port bind to a service using https, we might
write \mintinline{java}{Location: "socket://mult.example.com:443"} and
\mintinline{java}{Protocol: https}. The input port at the multiplication processor
would also have to match this, to ensure it runs on the correct port and speaks
the correct protocol.

It is a similar to bind an output port to an embedded service. However this is
done by setting the \mintinline{java}{Location} or \mintinline{java}{Protocol}
attributes. We must instead instruct the engine to embed the service, which
most importantly requires us to point to some executable service. The Jolie
engine supports several language for these embedded services, including Jolie,
Java, and JavaScript. For our desired deployment, we wanted to embed the
addition service inside of the calculator service. Assuming that the addition
service is written as a Jolie service, and its service is implemented in
\mintinline{text}{addition.ol}, then we may create an embedding as shown in
Listing \ref{lst:simple_embedding}. Just like in the case of the external
services, the input port of the receiving service \emph{must match}. In the
case of embedded service there must be an input port listining on the
\mintinline{java}{"local"} location.

Quite often the location of an input port is considered a deployment problem.
We see this quite clearly in the case of embedding a service. All current
solutions in Jolie, require us to \emph{modify the source code of a service},
simply to change where the service should listen. Best practices in Jolie
attempt to make this less of a problem by including a configuration file which
contains constants. The addition service, might include a file called
\verb!addition.iol! with constants setting up the location and protocol
of the service. An example of this is shown in Listing
\ref{lst:include_as_conf}.

\begin{listing}[H]
\begin{multicols}{2}

\begin{minted}[breaklines,fontsize=\footnotesize]{java}
// addition.ol
include "addition.iol"

inputPort Addition {
    Location: ADDIITON_LOCATION
    Protocol: ADDIITON_PROTOCOL
}
\end{minted}

\columnbreak

\begin{minted}[breaklines,fontsize=\footnotesize]{java}
// addition.iol
constants {
    ADDIITON_LOCATION = "local"
    ADDIITON_PROTOCOL = sodep
}
\end{minted}

\end{multicols}

\caption{A common Jolie practice for solving configuration of a service, is to
    include a file containing constants with the desired configuration.}

\label{lst:include_as_conf}

\end{listing}

\begin{observation}
    TODO Include source as configuration (via constants)
\end{observation}

\begin{listing}[H]
\begin{minted}{java}
embedded {
    Jolie:
        "addition.ol" in Addition
}
\end{minted}

\caption{Embedding the \mintinline{java}{addition} service in the
    \mintinline{java}{Addition} output port}

\label{lst:simple_embedding}

\end{listing}

With the code from Listing \ref{lst:simple_start} where the output port
\mintinline{java}{Console} is defined. The output port points to an embedding
of the console service, and is included directly in the \verb!console.iol!
file. This is a fairly common pattern used in Jolie, especially for services
that work in a library-like fashion (i.e. not intended as a stand-alone
service). This pattern is used for almost every single service in the Jolie
standard library.

\begin{observation}
    TODO Some services can only be used embedded and use includes to setup
    their embedding
\end{observation}

With the output ports correctly configured, we may now implement the actual
business logic for our calculator. For completeness sake this might look like
shown in Listing \ref{lst:op_impl}.

\begin{listing}[H]
\begin{minted}{java}
[sum(request)(response) {
    total = 0;
    for (i = 0, i < #request.numbers, i++) {
        // Add current number with total, and store result in total
        add@Addition({
            .a = total,
            .b =  request.numbers[i]
        })(total)
    };
    response = total
}]
\end{minted}
\caption{Implementing the checkout operation}
\label{lst:op_impl}
\end{listing}

Finally we want to reflcet slightly on the file structure that this service
ended up having. In order to use external services, we had to include files
which contain these interfaces. This version is worse when an embedding is
desired, since the entire service with its implementation is suddenly required.

The files from these external service are also hard to manage. It isn't
possible to simply move the source code of a service into its own directory.
This is not possible since includes, unlike in most other languages, are not
relative to the file performing the include, but rather relative to the project
level root. Thus if we wish to move the service to a new directory, all
includes in the source code would have to be updated. A common result of this
is that every single file gets dumped into the project level root. When all
files are stored in the same directory, we also get a much larger chance of
having name conflicts. As a result names tend to become rather large, to make
it less likely that a conflict occours.

The final file structure of the calculator service is shown in Listing
\ref{lst:file_structure}.

\begin{listing}[H]
\begin{minted}{text}
.
+-- include
|   +-- calculator.iol
|-- addition.iol
|-- addition.ol
|-- multiplication.iol
+-- calculator.ol
\end{minted}
\caption{File structure of the calculator service}
\label{lst:file_structure}
\end{listing}


\begin{observation}
    TODO File inclusion, littering of namespace. Causes very defensive naming
\end{observation}
