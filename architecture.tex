% Status: Draft

\section{Architecture}

% At this point we have covered the responsibilities of JPM. Here we will be
% discussing the actual technical details of JPM. As well as taking a deeper
% look into features provided, and why they are useful.

The entirety of the ecosystem around JPM is written in Jolie, using the
features that the Jolie Module System provides, along with the features that
JPM itself provides.

At the ten-thousand foot view of the architecture, it consists of three
core services, as shown in figure \ref{fig:high_level_arch}:

\begin{enumerate}

\item \textbf{Registry}: Responsible for serving packages known to the
registry.

\item \textbf{JPM}: Provides the back-end of JPM. This includes communication
with one, or more, registries, for example to download packages.

\item \textbf{CLI}: Provides the front-end of JPM. The front-end is responsible
for displaying a user-facing interface, and will communicate with the back-end
to perform the actual work.

\end{enumerate}

% TODO Replace with non-ascii
\begin{listing}[H]
\begin{minted}{text}
                +-----------+     +-------+     +------------+
                |  jpm-cli  | ==> |  jpm  | ==> |  registry  |
                +-----------+     +-------+     +------------+
\end{minted}
\caption{Ten-thousand foot view of the JPM architecture}
\label{fig:high_level_arch}
\end{listing}

