\section{Package Managers}

In this thesis we will work solely with application-level package managers.
As mentioned in the introduction, application-level package managers deal with
application-level packages. These packages are typically software components.

The work on the Jolie Package Manager (JPM), which this thesis presents, is
hugely inspired by the following works:

\begin{enumerate}
    \item \textbf{NPM:} The package manager for Node.js\autocite{NPMC}.
    \item \textbf{Yarn:} An alternative package manager for 
    Node.js\autocite{YARNC}.
    \item \textbf{Cargo:} Build tool for the Rust language\autocite{CARB}.
    \item \textbf{Gradle:} Build tool, typically used for JVM languages
    (e.g. Java)\autocite{GRAB}.
\end{enumerate}

In the remainder of this section we will cover the most common traits from
these works.

The primary role of a package manager, as the name would suggest, is to manage
software packages. A package manager usually describes software packages
through a special file, in this thesis we will refer to this file as the
package manifest. NPM and Yarn both use a file called \txtl{package.json},
Cargo uses a \txtl{Cargo.toml}, Gradle using a \txtl{build.gradle}.
Common for all of these are, that they typically describe the package itself,
and the dependencies of this package.

Listing \ref{lst:cargo_example} shows an example for Cargo, and shows a similar
manifest for NPM is shown in Listing \ref{lst:npm_example}. Common to both of
these examples, is that their descriptions use a few attributes for describing
the package itself. Additionally they have a section describing the
dependencies, i.e., the packages that should be ``managed''. Most importantly of
the attributes describing the package are the ``name'' and ``version'', which
can uniquely refer to a specific version of the package.

\begin{listing}[H]
\begin{minted}{ini}
[package]
name = "my-cargo-crate"
version = "1.2.3"
authors = ["Dan Sebastian Thrane <dathr12@student.sdu.dk>"]

[dependencies]
log = "0.3.8"
left-pad = "1.0.1"
\end{minted}
\caption{A simple manifest for a Cargo package (a ``Crate'')}
\label{lst:cargo_example}
\end{listing}


\begin{listing}[H]
\begin{minted}{json}
{
  "name": "my-npm-package",
  "version": "1.0.0",
  "description": "This is a NPM package",
  "main": "index.js",
  "scripts": {
    "test": "echo \"Error: no test specified\" && exit 1"
  },
  "author": "Dan Sebastian Thrane <dathr12@student.sdu.dk>",
  "license": "ISC",
  "dependencies": {
    "left-pad": "^1.1.3"
  }
}
\end{minted}
\caption{A simple manifest for a NPM package}
\label{lst:npm_example}
\end{listing}

Most package managing tools come with CLI (Command-Line Interface) tools, which
help the user perform various tasks. For example, both of the manifest examples
(Listing \ref{lst:cargo_example}, and \ref{lst:npm_example}) could be generated
via their respective tools as shown in Listing \ref{lst:tools_example}.

\begin{listing}[H]
\begin{minted}{text}
dan@host:/ # cargo new my-cargo-package
    Created library `my-cargo-package` project

dan@host:/ # npm init
This utility will walk you through creating a package.json file.
It only covers the most common items and tries to guess sensible defaults.

( ... Cut for brevity ... )
\end{minted}
\caption{Package managers typically ship CLI tools which help with common
    tasks}
\label{lst:tools_example}
\end{listing}

Similarly these tools can be used to download and install their dependencies,
for Cargo \txtl{cargo build}, for NPM \txtl{npm install}. The
packages, that these dependencies resolve to, can be stored in
various places. A common approach, which we see both NPM, Yarn, and Cargo, is
to use a default centralized registry. Gradle uses a similar approach but
doesn't have a single default registry (which they refer to as repositories).
The registries will act as a database of packages, which the tools will use
to download, and install packages. Registries typically also have some type of
user management. Such a feature becomes relevant when controlling who is
allowed to publish updates for a package.

The reader may have observed already, that we have been referring to both Cargo
and Gradle as package managers, even though they were listed as build tools.
The build tool description is pulled directly from their own descriptions.
However, the border between package managers and build tools are typically
blurry. Especially because many modern tools include dependency management as a
core part of their features. Even tools, like NPM, which advertise themselves
still build provide ``build tool''-like features, such as running tests and
starting the application. In this thesis, we will simply refer to our own tool,
the Jolie Package Manager, as a package manager. This is done, since
it is its primary responsibility, and technically there is nothing to
build in Jolie, due to its dynamic nature.
