\section{Microservices}

In this section we will summarize the concept of microservices. This summary is
mostly based on \autocite{MFMS}.

Microservices are most easily explained by comparing them to, perhaps a more
traditional architecture of, monoliths. A monolithic application is typically
built as a single unit. This unit will contain all the components of which it
is built.

Increasingly the industry has been feeling frustrations towards monolithic
applications.  Changes made to a monolith usually requires the entire
application to be redeployed.  Over time it may also become increasingly hard
to keep the boundaries between components clean.  Scaling of monoliths is also
of concern, as it typically requires the entire application to be scaled, as
opposed to just the components that need additional resources.

These frustrations led to a new architectural style, eventually named
microservices. Microservices attempt to tackle each of these problems, but with
a new architecture also comes new problems.

At its core, a microservice architecture will consist of many individual and
independently deployable services. The fact that services must be designed to
be independently deployable can make it significantly easier to scale.  The
services of a system will communicate with each other through message passing.
There are no real restrictions on how services should communicate with each
other. However, in practice services will typically communicate via the
network, typically using ``dumb'' protocols, such as HTTP.

Microservices typically have a high focus on their own published interfaces.
This comes, almost automatically, from the fact that they to communicate via
message passing.

% TODO I'm not sure if we need more here.
